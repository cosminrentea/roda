
\section{Scheduler / Cron}

API pentru sistemul de programare a executiei task-urilor (Cron / Scheduler). Task-urile sunt implementari ale interfetelor Runnable sau RunnableFuture. \subsection*{GET: /admin/scheduler/tasks}

\paragraph{Parametri de intrare si iesire:}
\begin{itemize}
\item \textbf{classname}
 Numele complet al clasei Java care implementeaza task-ul
\item\textbf{cron}
 String-ul specific in format Cron, ce permite definirea de regulii de programare a executiilor, si este format din campuri pentru: \begin{itemize}
\item 1.Seconds
\item 2.Minutes
\item 3.Hours
\item 4.Day-of-Month
\item 5.Month
\item 6.Day-of-Week
\item 7.Year (optional field)
 \end{itemize}
\item \textbf{enabled}
 daca task-ul este activat (sau dezactivat)
\item \textbf{timestamp\_next\_execution}
 - data si ora la care Task-ul este planificat pentru a fi lansat in executie (momentul planificat al urmatoarei executii)
\item \textbf{type}
: "scheduled"/"manual" (daca executia a fost programata sau a fost declansata manual)
\item \textbf{timestamp\_start}
 Momentul la care a inceput aceasta executie a task-ului
\item \textbf{result}
 rezultatul executiei (numar intreg, unde 0 reprezinta succes)
\item \textbf{duration}
 durata acestei executii (in milisecunde)
\item \textbf{stacktrace}
 detalii in caz de executie fara succes
\item \textbf{execute\_now}
 daca se declanseaza o executie a task-ului chiar in momentul update-ului curent
\item \textbf{execution\_result}
 rezultatul unei executii (folosit pentru obtinerea unei liste filtrate)
\item \textbf{timestamp\_begin}
 timestamp (folosit pentru obtinerea unei liste filtrate)
 \end{itemize}
 \subsection*{GET: /admin/scheduler/executions}

 \subsection*{GET: /admin/scheduler/executionsbytask/\{id\}}

 \subsection*{GET: /admin/scheduler/executions-full}

 \subsection*{GET: /admin/scheduler/tasks/\{id\}}

 \subsection*{GET: /admin/scheduler/executions/\{id\}}

 \subsection*{GET: /admin/scheduler/executions/filter-timestamp/\{timestamp\_begin\}}

 \subsection*{GET: /admin/scheduler/executions/filter-result/\{execution\_result\}}

 \subsection*{GET: /admin/scheduler/executions/filter-timestamp/\{timestamp\}}

 \subsection*{POST: /admin/scheduler/task-update}

 

\section{Audit}

API pentru componenta de audit a aplicatiei.  \subsection*{GET: /admin/audit/revisioninfo/\{revision\}}

\paragraph{Parametri}
\begin{itemize}
\item \textbf{\{revision\}}
 revizia ale carei informatii de audit sunt obtinute
 \end{itemize}
\paragraph{Returneaza:}
\begin{itemize}
\item \textbf{timestamp}
 Data si ora modificarii
\item \textbf{revision}
 Versiunea (revizia) modificarii curente
\item \textbf{username}
 Numele utilizatorului care a facut modificarea
\item \textbf{userid}
 Id-ul utilizatorului care a facut modificarea
\item \textbf{modtype}
 Tipul modificarii (update, delete, insert)
\item \textbf{nrobjects}
 Numarul de tabele afectate de modificare
\item\textbf{objects}
 Obiectele (tabelele) modificate \begin{itemize}
\item \textbf{objectname}
 Numele tabelului
\item \textbf{nrrows}
 Numarul de inregistrari modificate in cadrul tabelului, la revizia respectiva
\item\textbf{rows}
 Inregistrarile modificate la revizia respectiva \begin{itemize}
\item \textbf{indice}
 Cheia primara a inregistrarii modificate
\item \textbf{modtype}
 Tipul modificarii (insert, update, delete)
\item \textbf{nrfields}
 Numarul de campuri afectate de modificare 
\item\textbf{auditfields}
 Cheie multipla pentru campurile modificate \begin{itemize}
\item \textbf{auditfield}
 Numele campului
\item \textbf{auditvalue}
 Valoarea campului la revizia curenta
 \end{itemize}
 \end{itemize}
 \end{itemize}
 \end{itemize}
 \subsection*{GET: /admin/audit/revisions-by-object/\{object\}/\{id\}}

\paragraph{Parametri}
\begin{itemize}
\item \textbf{\{object\}}
 tabelul din care face parte
\item \textbf{\{id\}}
 id-ul elementului pentru care se cer elementele de audit
 \end{itemize}
\paragraph{Returneaza:}
\begin{itemize}
\item \textbf{object}
 Numele tabelului din care face parte inregistrarea pentru care sunt solicitate informatiile de audit
\item \textbf{indice}
 Cheia primara a inregistrarii
\item \textbf{nrrev}
 Numarul de revizii care au afectat elementul \{id\} din tabelul \{object\}
\item\textbf{revisions}
 Reviziile care au afectat elementul \{id\} din tabelul \{object\} \begin{itemize}
\item \textbf{revision}
 Versiunea (revizia) corespunzatoare modificarii
\item \textbf{timestamp}
 Data si ora modificarii
\item \textbf{username}
 Numele utilizatorului care a efectuat revizia
\item \textbf{userid}
 Id-ul utilizatorului care a efectuat revizia
\item \textbf{modtype}
 Tipul modificarii (update, delete, insert)
\item\textbf{nrfields}
 Numarul de campuri afectate de modificare \begin{itemize}
\item\textbf{auditfields}
 Cheie multipla pentru campurile modificate \begin{itemize}
\item \textbf{auditfield}
 Numele campului
\item \textbf{auditvalue}
 Valoarea campului la revizia curenta
 \end{itemize}
 \end{itemize}
 \end{itemize}
 \end{itemize}
 \subsection*{GET: /admin/audit/revisions-after-timestamp/\{timestamp\}}

\paragraph{Parametri}
\begin{itemize}
\item \textbf{\{timestamp\}}
 data si ora dupa care au avut loc reviziile ale caror informatii de audit sunt obtinute
 \end{itemize}
\paragraph{Returneaza:}
\begin{itemize}
\item \textbf{timestamp}
 Data si ora modificarii
\item \textbf{revision}
 Versiunea (revizia) modificarii curente
\item \textbf{username}
 Numele utilizatorului care a facut modificarea
\item \textbf{userid}
 Id-ul utilizatorului care a facut modificarea
\item \textbf{nrobjects}
 Numarul de tabele afectate de modificare
 \end{itemize}
 \subsection*{GET: /admin/audit/revisions-by-user/\{username\}/\{object\}}

\paragraph{Parametri}
\begin{itemize}
\item \textbf{\{username\}}
 numele utilizatorului
\item \textbf{\{object\}}
 obiectul (tabelul) modificat de catre utilizator
 \end{itemize}
\paragraph{Returneaza:}
\begin{itemize}
\item \textbf{username}
 Numele utilizatorului care a facut modificarea
\item \textbf{userid}
 Id-ul utilizatorului care a facut modificarea
\item \textbf{object}
 Tabelul modificat de catre utilizator
\item \textbf{nrrev}
 Numarul de revizii efectuate de catre utilizator asupra tabelului
\item\textbf{revisions}
 Reviziile efectuate de catre utilizator asupra tabelului \begin{itemize}
\item \textbf{revision}
 Versiunea (revizia) corespunzatoare modificarii 
\item \textbf{timestamp}
 Data si ora reviziei
\item \textbf{nrrows}
 Numarul de inregistrari modificate in cadrul reviziei
 \end{itemize}
 \end{itemize}
 

\section{CMS Dashboard}

API pentru componenta Dashboard a sistemului CMS. Aceasta componenta va fi dezvoltata ulterior celorlalte; pentru moment, introducem doar placeholders pentru metodele corespunzatoare. 

\section{CMS Layout Management}

API pentru modulul CMS, submodulul Layouts. Un layout este un fragment de cod HTML care poate contine una sau mai multe referinte catre snippet-uri sau fisiere. In functie de cum este realizata implementarea designului, layout-ul poate fi foarte simplu sau foarte complicat. La nivel de interfata de administrare, este important faptul ca un layout este un fragment de text. Odata implementat site-ul public, numarul de layout-uri ale acestuia va fi egal cu numarul de pagini diferite care vor exista. \subsection*{GET: /admin/layout/layoutlist}

\paragraph{Returneaza:}
\begin{itemize}
\item \textbf{name}
 Numele layout-ului (fara cale)
\item \textbf{id}
 Id-ul layout-ului din tabelul din baza de date
\item \textbf{groupid}
 Id-ul grupului din care face parte layout-ul
\item \textbf{itemtype}
 Tipul inregistrarii: layout
\item \textbf{directory}
 Calea (grupurile concatenate) din care face parte layout-ul.
\item \textbf{description}
 Scurta descriere a layout-ului<br />
\item \textbf{nrpagesusing}
 Numarul de pagini care folosesc acest layout
 \end{itemize}
 \subsection*{GET: /admin/layout/layoutinfo/\{id\}}

\paragraph{Returneaza:}
\begin{itemize}
\item \textbf{name}
 Numele layout-ului din baza de date
\item \textbf{id}
 Id-ul layout-ului din baza de date
\item \textbf{description}
 Descrierea layout-ului respectiv
\item \textbf{groupid}
 Id-ul grupului din care face parte layout-ul
\item \textbf{itemtype}
 Tipul inregistrarii (layout)
\item \textbf{directory}
 Calea (grupurile concatenate) din care face parte layout-ul.
\item \textbf{pagesnumber}
 Numarul de pagini care folosesc acest layout
\item \textit{layoutusage}
 Paginile care utilizeaza layout-ul. \begin{itemize}
\item \textbf{id}
 Id-ul paginii respective
\item \textbf{name}
 Numele paginii 
\item \textbf{url}
 Url-ul paginii (specificat in tabel)
\item \textbf{lang}
 Limba in care este scrisa pagina
\item \textbf{visible}
 Pagina este sau nu vizibila
\item \textbf{pagetypeid}
 Id-ul tipului de pagina din tabelul cms\_page\_type
\item \textbf{pagetypename}
 Numele tipului de pagina corespunzator
 \end{itemize}
 \end{itemize}
 \subsection*{GET: /admin/layout/groupinfo/\{id\}}

\begin{itemize}
\item \textbf{name}
 Numele elementului (layout sau grup)
\item \textbf{id}
 Id-ul din baza de date. 
\item \textbf{groupid}
 Grupul din care face parte grupul curent 
\item \textbf{directory}
 Calea catre grupul curent (de exemplu, Main/Secondary Layouts)
\item \textbf{itemtype}
 Tipul inregistrarii (layoutgroup)
\item \textbf{leaf}
 True, daca grupul este gol
\item \textbf{icon-cls}
 Determina tipul de pictograma folosita la nodul respectiv 
\item \textbf{description}
 Descrierea grupului (din baza de date)
\item \textbf{layoutsnumber}
 Numarul de layout-uri din acest grup
 \end{itemize}
 \subsection*{GET: /admin/layout/layouttree/}

\paragraph{Returneaza o structura de date recursiva, cu urmatoarele proprietati:}
\begin{itemize}
\item \textbf{name}
 Numele elementului (layout sau grup)
\item \textbf{indice}
 Id-ul din baza de date. 
\item \textbf{expanded}
 True sau false; va fi utilizat atunci cand va fi folosita sesiunea, pentru a indica daca arborescenta apare sau nu desfasurata
\item \textbf{itemtype}
 Tipul inregistrarii (layout sau layoutgroup)
\item \textbf{children}
 referinta catre un array de obiecte cu aceeasi structura care reprezinta copiii.
\item \textbf{leaf}
 True, daca elementul este un grup gol sau un layout
\item \textbf{icon-cls}
 Determina tipul de pictograma folosita la nodul respectiv
 \end{itemize}
 \subsection*{GET: /admin/layout/grouptree}

\paragraph{Returneaza o structura de date recursiva, cu urmatoarele proprietati:}
\begin{itemize}
\item \textbf{name}
 Numele elementului
\item \textbf{indice}
 Id-ul din baza de date
\item \textbf{expanded}
 True sau false; va fi utilizat atunci cand va fi folosita sesiunea, pentru a indica daca arborescenta apare sau nu desfasurata 
\item \textbf{children}
 Referinta catre un array de obiecte cu aceeasi structura care reprezinta copiii
\item \textbf{leaf}
 True, daca grupul este gol
\item \textbf{icon-cls}
 Determina tipul de pictograma folosita la nodul respectiv
 \end{itemize}
 \subsection*{POST: /admin/layout/groupsave}

\paragraph{Parametri:}
\begin{itemize}
\item \textbf{groupname}
 Numele noului grup
\item \textbf{parent}
 Id-ul parintelui grupului nou creat
\item \textbf{description}
 Descrierea grupului respectiv
 \end{itemize}
 \subsection*{POST: /admin/layout/groupempty}

\paragraph{Parametri:}
\begin{itemize}
\item \textbf{groupid}
 Id-ul grupului al carui continut trebuie sters
 \end{itemize}
 \subsection*{POST: /admin/layout/groupdrop}

\paragraph{Parametri:}
\begin{itemize}
\item \textbf{grupid}
 Id-ul grupului care trebuie sters
 \end{itemize}
 \subsection*{POST: /admin/layout/layoutdrop}

\paragraph{Parametri:}
\begin{itemize}
\item \textbf{layoutid}
 Id-ul layout-ului care trebuie sters
 \end{itemize}
 \subsection*{POST: /admin/layout/layoutsave}

\paragraph{Parametri:}
\begin{itemize}
\item \textbf{group}
 Grupul in care trebuie adaugat layout-ul
\item \textbf{content}
 Continutul layout-ului respectiv
\item \textbf{name}
 Numele layout-ului
\item \textbf{id}
 Parametru optional; daca nu exista, inseamna ca trebuie creat un layout nou, altfel se modifica layout-ul existent
 \end{itemize}
 \subsection*{POST: /admin/layout/move}

\paragraph{Parametri:}
\begin{itemize}
\item \textbf{group}
 Grupul in care trebuie pus layoutul
\item \textbf{id}
 Id-ul layoutului care trebuie mutat
 \end{itemize}
 \subsection*{POST: /admin/layout/groupmove}

\paragraph{Parametri:}
\begin{itemize}
\item \textbf{parent}
 Grupul in care trebuie pus grupul (noul parinte)
\item \textbf{group}
 Id-ul grupului care trebuie mutat
 \end{itemize}
 

\section{CMS Snippet Management}

API pentru modulul CMS, submodulul Snippets. Un snippet este un fragment refolosibil de text, cod HTML, Javascript etc., care poate contine una sau mai multe referinte catre alte snippet-uri sau fisiere. Snippet-urile sunt grupate in grupuri care alcatuiesc o arborescenta. \subsection*{GET: /admin/snippet/snippetlist}

\paragraph{Returneaza un array de elemente avand urmatoarea structura:}
\begin{itemize}
\item \textbf{name}
 Numele snippet-ului
\item \textbf{id}
 Id-ul snippet-ului
\item \textbf{groupid}
 Id-ul grupului din care face parte snippet-ul
\item \textbf{directory}
 Calea (grupurile concatenate) din care face parte snippet-ul.
\item \textbf{content}
 Primele \{n\} caractere din snippet, cu javascript, html filtrate (de exemplu, tag-urile html vor fi transformate cu ajutorul \&gt; si \&lt;). N va fi configurabil
 \end{itemize}
 \subsection*{GET: /admin/snippet/snippetinfo/\{id\}}

\paragraph{Returneaza:}
\begin{itemize}
\item \textbf{name}
 Numele snippet-ului din baza de date
\item \textbf{id}
 Id-ul snippet-ului din baza de date
\item \textbf{content}
 Continutul snippet-ului respectiv
\item \textbf{groupid}
 Id-ul grupului din care face parte snippet-ul
\item \textbf{directory}
 Calea (grupurile concatenate) din care face parte snippet-ul
\item\textbf{snippetusage}
 Elementele care utilizeaza snippet-ul (array) \begin{itemize}
\item \textbf{id}
 Id-ul elementului respectiv
\item \textbf{name}
 Numele elementului respectiv
\item \textbf{type}
 Tipul elementului respectiv (pagina, layout, alt snippet)
 \end{itemize}
 \end{itemize}
 \subsection*{GET: /admin/snippet/groupinfo/\{id\}}

\begin{itemize}
\item \textbf{name}
 Numele elementului (layout sau grup)
\item \textbf{id}
 Id-ul din baza de date. 
\item \textbf{groupid}
 Grupul din care face parte grupul curent 
\item \textbf{directory}
 Calea catre grupul curent (Main/First snippets)
\item \textbf{itemtype}
 Tipul inregistrarii (snippetgroup)
\item \textbf{leaf}
 True, daca grupul de snippet-uri este gol
\item \textbf{icon-cls}
 Determina tipul de pictograma folosita la nodul respectiv.<br />
\item \textbf{description}
 Descrierea grupului (din baza de date)
\item \textbf{snippetsnumber}
 Numarul de snippet-uri din acest grup
 \end{itemize}
 \subsection*{GET: /admin/snippet/snippettree}

\paragraph{Returneaza o structura de date recursiva, cu urmatoarele proprietati:}
\begin{itemize}
\item \textbf{name}
 Numele elementului (snippet sau grup)
\item \textbf{indice}
 Id-ul din baza de date
\item \textbf{expanded}
 True sau false; va fi utilizat atunci cand va fi folosita sesiunea, pentru a indica daca arborescenta apare sau nu desfasurata 
\item \textbf{children}
 Referinta catre un array de obiecte cu aceeasi structura care reprezinta copiii.
\item \textbf{leaf}
 True, daca grupul este vid
\item \textbf{icon-cls}
 Determina tipul de pictograma folosita la nodul respectiv
\item \textbf{content}
 Fragment de continut (200 de caractere, configurabil)
 \end{itemize}
 \subsection*{GET: /admin/snippet/grouptree}

\paragraph{Returneaza o structura de date recursiva, cu urmatoarele proprietati:}
\begin{itemize}
\item \textbf{name}
 Numele elementului
\item \textbf{indice}
 Id-ul din baza de date
\item \textbf{expanded}
 True sau false; va fi utilizat atunci cand va fi folosita sesiunea, pentru a indica daca arborescenta apare sau nu desfasurata 
\item \textbf{children}
 Referinta catre un array de obiecte cu aceeasi structura care reprezinta copiii.
\item \textbf{leaf}
 True, daca elementul este snippet sau grup vid
\item \textbf{icon-cls}
 Determina tipul de pictograma folosita la nodul respectiv
 \end{itemize}
 \subsection*{POST: /admin/snippet/groupsave}

\paragraph{Parametri:}
\begin{itemize}
\item \textbf{groupname}
 Numele noului grup
\item \textbf{parent}
 Id-ul parintelui grupului nou creat
\item \textbf{description}
 Descrierea grupului respectiv
 \end{itemize}
 \subsection*{POST: /admin/snippet/groupempty}

\paragraph{Parametri:}
\begin{itemize}
\item \textbf{groupid}
 Id-ul grupului al carui continut trebuie sters
 \end{itemize}
 \subsection*{POST: /admin/snippet/groupdrop}

\paragraph{Parametri:}
\begin{itemize}
\item \textbf{grupid}
 Id-ul grupului de snippet-uri care trebuie sters
 \end{itemize}
 \subsection*{POST: /admin/snippet/snippetdrop}

\paragraph{Parametri:}
\begin{itemize}
\item \textbf{snippetid}
 Id-ul snippet-ului care trebuie sters
 \end{itemize}
 \subsection*{POST: /admin/snippet/snippetsave}

\paragraph{Parametri:}
\begin{itemize}
\item \textbf{group}
 Grupul in care trebuie pus snippet-ul
\item \textbf{content}
 Continutul snippet-ului respectiv
\item \textbf{name}
 Numele snippet-ului
\item \textbf{id}
 Parametru optional; daca nu exista, va fi creat un snippet nou, altfel modificam snippet-ul existent
 \end{itemize}
 \subsection*{POST: /admin/snippet/move}

\paragraph{Parametri:}
\begin{itemize}
\item \textbf{group}
 Grupul in care trebuie pus snippet-ul
\item \textbf{id}
 Id-ul snippet-ului care trebuie mutat
 \end{itemize}
 \subsection*{POST: /admin/snippet/groupmove}

\paragraph{Parametri:}
\begin{itemize}
\item \textbf{parent}
 Grupul in care trebuie pus grupul (noul parinte)
\item \textbf{group}
 Id-ul grupului care trebuie mutat
 \end{itemize}
 

\section{CMS File Management}

API pentru modulul CMS, submodulul Files \subsection*{GET: /admin/file/filegrid}

\paragraph{Returneaza:}
\begin{itemize}
\item \textbf{name}
 Numele fisierului (fara cale)
\item \textbf{alias}
 Alias-ul fisierului respectiv
\item \textbf{filesize}
 Dimensiunea fisierului, din baza de date
\item \textbf{id}
 Id-ul fisierului din tabelul din baza de date
\item \textbf{filetype}
 Clasa generala din care face parte fisierul
\item \textbf{folderid}
 Id-ul folderului din care face parte fisierul
\item \textbf{directory}
 Calea (directorul) din care face parte fisierul (obtinuta din baza de date).
 \end{itemize}
 \subsection*{GET: /admin/cms/fileinfo/\{id\}}

\paragraph{Returneaza:}
\begin{itemize}
\item \textbf{filename}
 Numele fisierului din baza de date
\item \textbf{filetype}
 Tipul fisierului
\item \textbf{id}
 Id-ul fisierului din baza de date
\item \textbf{filesize}
 Dimensiunea fisierului, exprimata in kb
\item \textbf{filepath}
 Calea reala a fisierului pe sistemul de fisiere
\item \textbf{alias}
 Aliasul fisierului
\item \textbf{fileurl}
 Url-ul cu care poate fi accesat fisierul
\item\textbf{fileproperties}
 Proprietatile fisierului din tabelul de proprietati. Acestea sunt perechi cheie - valoare care depind de tipul fisierului \begin{itemize}
\item \textbf{name}
 Numele proprietatii (cheia)
\item \textbf{value}
 Valoarea proprietatii
 \end{itemize}
\item\textbf{fileusage}
 Componenetele in care este utilizat fisierul. Fisierele vor fi specificate prin alias cu ajutorul unei expresii de tip macro (de exemplu, \{\{File:alias\}\}) in corpul componentelor de continut. \begin{itemize}
\item \textbf{type}
 Tipul de element de continut in care este utilizat fisierul (layout, snippet, etc.)
\item \textbf{text}
 Numele elementului respectiv<br />
\item \textbf{id}
 Id-ul elementului
 \end{itemize}
 \end{itemize}
 \subsection*{GET: /admin/file/folderinfo/\{id\}}

\begin{itemize}
\item \textbf{name}
 Numele folderului
\item \textbf{id}
 Id-ul din baza de date. 
\item \textbf{groupid}
 Folderul din care face parte folderul curent 
\item \textbf{directory}
 Calea catre folderul curent (de exemplu, Main/Secondary Layouts)
\item \textbf{itemtype}
 Tipul inregistrarii (folder)
\item \textbf{filesnumber}
 Numarul de fisiere din acest grup
 \end{itemize}
 \subsection*{GET: /admin/file/filetree/}

\paragraph{Returneaza o structura de date recursiva, cu urmatoarele proprietati:}
\begin{itemize}
\item \textbf{name}
 Numele elementului
\item \textbf{indice}
 Id-ul din baza de date
\item \textbf{expanded}
 True sau false; va fi utilizat atunci cand va fi folosita sesiunea, pentru a indica daca arborescenta apare sau nu desfasurata 
\item \textbf{children}
 Referinta catre un array de obiecte cu aceeasi structura care reprezinta copiii.
\item \textbf{alias}
 Alias-ul fisierului (valabil doar pentru intrarile care sunt fisiere)
\item \textbf{leaf}
 True, daca elementul este un fisier sau un folder gol
\item \textbf{icon-cls}
 Determina tipul de pictograma folosita la nodul respectiv
\item \textbf{filesize}
 Dimensiunea fisierului (doar pentru nodurile care sunt fisiere).
 \end{itemize}
 \subsection*{GET: /admin/file/foldertree}

\paragraph{Returneaza o structura de date recursiva, cu urmatoarele proprietati:}
\begin{itemize}
\item \textbf{name}
 Numele elementului
\item \textbf{indice}
 Id-ul din baza de date
\item \textbf{expanded}
 True sau false; va fi utilizat atunci cand va fi folosita sesiunea, pentru a indica daca arborescenta apare sau nu desfasurata 
\item \textbf{children}
 Referinta catre un array de obiecte cu aceeasi structura care reprezinta copiii
\item \textbf{alias}
 Alias-ul fisierului (valabil doar pentru intrarile care sunt fisiere)
\item \textbf{leaf}
 True, daca elementul este un fisier sau un folder gol
\item \textbf{icon-cls}
 Determina tipul de pictograma folosita la nodul respectiv
\item \textbf{filesize}
 Dimensiunea fisierului (doar pentru nodurile care sunt fisiere).
 \end{itemize}
 \subsection*{POST: /admin/file/foldersave}

\paragraph{Parametri:}
\begin{itemize}
\item \textbf{foldername}
 Numele noului folder
\item \textbf{parent}
 Id-ul parintelui folderului nou creat
\item \textbf{description}
 Descrierea folderului respectiv
 \end{itemize}
 \subsection*{POST: /admin/file/folderempty}

\paragraph{Parametri:}
\begin{itemize}
\item \textbf{folderid}
 Id-ul folderului care trebuie sters
 \end{itemize}
 \subsection*{POST: /admin/file/folderdrop}

\paragraph{Parametri:}
\begin{itemize}
\item \textbf{folderid}
 Id-ul folderului care trebuie sters
 \end{itemize}
 \subsection*{POST: /admin/file/drop}

\paragraph{Parametri:}
\begin{itemize}
\item \textbf{fileid}
 Id-ul fisierului care trebuie sters
 \end{itemize}
 \subsection*{POST: /admin/file/save}

\paragraph{Parametri:}
\begin{itemize}
\item \textbf{parent}
 Folderul in care trebuie pus fisierul
\item \textbf{fileAlias}
 Alias-ul fisierului respectiv
\item \textbf{id}
 Parametru optional; daca nu exista, va trebui creat un fisier nou, altfel se va modifica fisierul existent
\item \textbf{file}
 Fisierul respectiv - multipart
 \end{itemize}
 \subsection*{POST: /admin/file/move}

\paragraph{Parametri:}
\begin{itemize}
\item \textbf{folder}
 Folderul in care trebuie pus fisierul
\item \textbf{id}
 Id-ul fisierului care trebuie mutat
 \end{itemize}
 \subsection*{POST: /admin/layout/foldermove}

\paragraph{Parametri:}
\begin{itemize}
\item \textbf{parent}
 Folderul in care trebuie pus folderul dat (noul parinte)
\item \textbf{folder}
 Id-ul folderului care trebuie mutat
 \end{itemize}
 

\section{CMS Page Management}

API pentru modulul CMS, submodulul Pages \subsection*{GET: /admin/cmspagestree}

\paragraph{Returneaza:}
\begin{itemize}
\item \textbf{title}
 Titlul paginii
\item \textbf{indice}
 Id-ul paginii
\item \textbf{lang}
 Limba paginii
\item \textbf{menutitle}
 Titlul alternativ pentru meniuri. 
\item \textbf{synopsis}
 Camp de tip text in care va intra un scurt rezumat al paginii, cu scopul afisarii<br />
\item \textbf{target}
 none, \_blank, parent, \_self sau \_top
\item \textbf{url}
 Fragmentul de url al paginii
\item \textbf{default}
 True sau false; true inseamna ca pagina este implicita (cea care se afiseaza cand nu se solicita o pagina specifica)
\item \textbf{externalredirect}
 Full url. Daca este completat, pagina apelata va redirecta catre acest url 
\item \textbf{internalredirect}
 Internal url. Redirectare catre una dintre paginile interne
\item \textbf{layout}
 Layout-ul folosit de pagina curenta
\item \textbf{cacheable}
 Integer mai mare decat 0 daca pagina va fi tinuta in cache n minute
\item \textbf{published}
 True sau false; true, daca pagina este publicata
\item \textbf{pagetype}
 tipul de pagina (variante: content, error404)
 \end{itemize}
 \subsection*{GET: /admin/cmspageinfo/\{id\}}

\paragraph{Parametri:}
\begin{itemize}
\item \textbf{id}
 Id-ul paginii respective
 \end{itemize}
\paragraph{Returneaza:}
\begin{itemize}
\item \textbf{title}
 Titlul paginii
\item \textbf{id}
 Id-ul paginii
\item \textbf{lang}
 Limba paginii
\item \textbf{menutitle}
 Titlul alternativ pentru meniuri
\item \textbf{synopsis}
 Camp de tip text in care va intra un scurt rezumat al paginii, cu scopul afisarii<br />
\item \textbf{target}
 none, \_blank, parent, \_self sau \_top
\item \textbf{url}
 Fragmentul de url al paginii
\item \textbf{default}
 True sau false; true inseamna ca pagina este implicita (cea care se afiseaza cand nu se solicita o pagina specifica)
\item \textbf{externalredirect}
 Full url. Daca este completat, pagina apelata va redirecta catre acest url 
\item \textbf{internalredirect}
 Internal url. Redirectare catre una dintre paginile interne
\item \textbf{layout}
 Layout-ul folosit de pagina curenta (numele)
\item \textbf{layoutid}
 Id-ul layout-ului utilizat de pagina
\item \textbf{layoutdirectory}
 Grupurile din care face parte layout-ul, concatenate
\item \textbf{cacheable}
 Integer mai mare decat 0 daca pagina va fi tinuta in cache n minute
\item \textbf{published}
 True sau false; true, daca pagina este publicata
\item \textbf{pagetype}
 Tipul de pagina (variante: content, error404)
\item \textbf{parentid}
 Id-ul paginii supraordonate
\item \textbf{parenttitle}
 Titlul paginii supraordonate
\item \textbf{parentpath}
 Calea completa a paginii subordonate (obtinuta prin concatenarea numelor tuturor parintilor)
\item \textbf{content}
 continutul paginii
 \end{itemize}
 \subsection*{GET: /admin/cmspageaccess/\{id\}}

\paragraph{Parametri:}
\begin{itemize}
\item \textbf{timestamp}
 Data si ora la care a fost facuta solicitarea curenta
\item \textbf{ipaddr}
 Adresa de IP de la care a fost facuta solicitarea
\item \textbf{userid}
 Id-ul utilizatorului care a facut solicitarea, daca exista (solicitarile pot fi si anonime daca nu este autentificat)
\item \textbf{username}
 Numele utilizatorului care a facut autentificarea, daca exista
\item \textbf{referer}
 Pagina care a trimis catre pagina curenta
\item \textbf{useragent}
 Identificatorul programului folosit pentru navigatie
 \end{itemize}
 \subsection*{GET: /admin/cmspagepreview/\{id\}}

\paragraph{Returneaza:}
\begin{itemize}
\item \textbf{url}
 url-ul unde se gaseste copia temporara a paginii
 \end{itemize}
 \subsection*{POST: /admin/cmspagesave}

\paragraph{Parametri:}
\begin{itemize}
\item \textbf{title}
 Titlul paginii
\item \textbf{id}
 Id-ul paginii
\item \textbf{lang}
 Limba paginii
\item \textbf{menutitle}
 Titlul alternativ pentru meniuri. 
\item \textbf{synopsis}
 Camp de tip text in care va intra un scurt rezumat al paginii, cu scopul afisarii<br />
\item \textbf{target}
 none, \_blank, parent, \_self sau \_top
\item \textbf{url}
 Fragmentul de url al paginii
\item \textbf{default}
 True sau false; true inseamna ca pagina este default (cea care se afiseaza cand nu se solicita o pagina specifica)
\item \textbf{externalredirect}
 Full url. Daca este completat, pagina apelata va redirecta catre acest url 
\item \textbf{internalredirect}
 Internal url. Redirectare catre una dintre paginile interne
\item \textbf{layout}
 Layout-ul folosit de pagina curenta (id-ul)
\item \textbf{cacheable}
 Integer mai mare decat 0 daca pagina va fi tinuta in cache n minute
\item \textbf{published}
 True sau false; true, daca pagina este publicata
\item \textbf{pagetype}
 Tipul de pagina (variante: content, error404)
\item \textbf{parentid}
 Id-ul paginii supraordonate
\item \textbf{content}
 Continutul paginii
 \end{itemize}
 \subsection*{POST: /admin/cmspagemove}

\paragraph{Parametri:}
\begin{itemize}
\item \textbf{parent}
 Noua pagina supraordonata
\item \textbf{id}
 Id-ul paginii de mutat
 \end{itemize}
 \subsection*{POST: /admin/cmspagedelete}

\paragraph{Parametri}
\begin{itemize}
\item \textbf{id}
 id-ul paginii care trebuie stearsa. Nu se accepta decat stergerea unei pagini care nu are pagini subordonate.
 \end{itemize}
 

\section{USER Management}

API pentru modulul User ManagementModulul User Management face parte din interfata de administrare a arhivei RODA si se ocupa de gestiunea utilizatorilor acesteia, atat de administratorii sistemului care au acces la interfata de administrare cat si de utilizatorii care au acces la site-ul public si la arhiva de date.Administratorii sistemului vor fi cuprinsi in categoria "administrators" in timp ce utilizatorii normali vor fi cuprinsi in categoria "users". \subsection*{GET: /admin/grouplist}

\paragraph{Returneaza o lista de elemente, fiecare dintre ele caracterizate de urmatoarele atribute:}
\begin{itemize}
\item \textbf{id}
 Id-ul grupului corespunzator
\item \textbf{name}
 Numele grupului respectiv
\item \textbf{nrusers}
 Numarul de utilizatori din grupul respectiv
\item \textbf{description}
 Descrierea grupului respectiv
 \end{itemize}
 \subsection*{GET: /admin/groupinfo/\{id\}}

\paragraph{Returneaza un singur element, care contine urmatoarele:}
\begin{itemize}
\item \textbf{id}
 Id-ul grupului corespunzator
\item \textbf{name}
 Numele grupului respectiv
\item \textbf{nrusers}
 Numarul de utilizatori din grupul respectiv
\item \textbf{description}
 Descrierea grupului respectiv
 \end{itemize}
 \subsection*{GET: /admin/userslist/\{id\}}

\paragraph{Returneaza o colectie de structuri de tip utilizator, fiecare dintre ele continand urmatoarele:}
\begin{itemize}
\item \textbf{id}
 Id-ul utilizatorului respectiv
\item \textbf{username}
 Username-ul utilizatorului respectiv
\item \textbf{firstname}
 Prenumele utilizatorului respectiv
\item \textbf{lastname}
 Numele utilizatorului respectiv
\item \textbf{email}
 Adresa de email a utilizatorului
\item \textbf{enabled}
 True daca utilizatorul este activ, false daca este blocat. 
 \end{itemize}
 \subsection*{GET: /admin/usersbygroup/\{id\}}

\paragraph{Returneaza o colectie de structuri de tip utilizator, fiecare dintre ele continand urmatoarele:}
\begin{itemize}
\item \textbf{id}
 Id-ul utilizatorului respectiv
\item \textbf{username}
 Username-ul utilizatorului respectiv
\item \textbf{email}
 Adresa de email a utilizatorului
\item \textbf{enabled}
 True daca utilizatorul este activ, false daca este blocat. 
 \end{itemize}
 \subsection*{GET: /admin/userinfo/\{id\}}

\paragraph{Returneaza datele unui singur utilizator, impreuna cu o serie de colectii adiacente (profil, activitate, setari).}
\begin{itemize}
\item \textbf{id}
 Id-ul utilizatorului respectiv
\item \textbf{username}
 Username-ul utilizatorului respectiv
\item \textbf{email}
 Adresa de email a utilizatorului
\item \textbf{enabled}
 True daca utilizatorul este activ, false daca este blocat. 
\item\textbf{groups}
 Grupurile din care face parte utilizatorul. Structura identica cu cea de la groupinfo \begin{itemize}
\item \textbf{id}
 Id-ul grupului corespunzator
\item \textbf{name}
 Numele grupului respectiv
\item \textbf{nrusers}
 Numarul de utilizatori din grupul respectiv
\item \textbf{description}
 Descrierea grupului respectiv
 \end{itemize}
\item\textbf{profile}
 Elementele profilului utilizatorului \begin{itemize}
\item \textbf{salutation}
 domnul, doamna, domnisoara, in functie de caz
\item \textbf{firstname}
 Prenumele
\item \textbf{lastname}
 Numele
\item \textbf{middlename}
 Alt prenume 
\item \textbf{title}
 Titlul, daca este cazul
\item \textbf{image}
 Fotografia
\item \textbf{address1}
 Adresa
\item \textbf{address2}
 Adresa
\item \textbf{city}
 Orasul
\item \textbf{country}
 Tara utilizatorului
\item \textbf{phone}
 nNmarul de telefon al utilizatorului
\item \textbf{sex}
 Sexul utilizatorului
\item \textbf{birthdate}
 Data nasterii utilizatorului (este posibil sa existe date care nu sunt accesibile decat utilizatorilor peste 18 ani)
 \end{itemize}
\item\textbf{lastmessages}
 Ultimele 10 mesaje trimise sau primite de utilizator pe site-ul RODA \begin{itemize}
\item \textbf{timestamp}
 Data si ora mesajului
\item \textbf{subject}
 Subiectul mesajului
\item \textbf{direction}
 Trimis sau primit
\item \textbf{read}
 True daca mesajul a fost citit, false daca nu
\item \textbf{message}
 Corpul mesajului
 \end{itemize}
\item\textbf{activity}
 Ultimele 50 de operatii pe care utilizatorul le-a executat pe sitepul RODA \begin{itemize}
\item \textbf{timestamp}
 Data si ora activitatii
\item \textbf{type}
 Tipul activitatii (login, logout, download, vizionare date, etc.)
\item \textbf{details}
 Tn functie de activitate este posibil sa existe detalii, de exemplu la autentificare se vor inregistra IP-ul de pe care a fost accesat site-ul, tipul de navigator, sistemul de operare
\item \textbf{error}
 True sau false daca activitatea respectiva a produs o eroare
\item \textbf{errormessage}
 Mesajul de eroare
\item \textbf{errordetails}
 Detalii ale erorii, daca exista
 \end{itemize}
 \end{itemize}
 \subsection*{GET: /admin/usermessages/\{id\}}

\paragraph{Returneaza toate mesajele trimise sau primite de utilizator, in sistem paginat}
\begin{itemize}
\item \textbf{timestamp}
 data si ora mesajului
\item \textbf{subject}
 subiectul mesajului
\item \textbf{direction}
 trimis sau primit
\item \textbf{read}
 true daca mesajul a fost citit, false daca nu
\item \textbf{message}
 corpul mesajului
 \end{itemize}
 \subsection*{GET: /admin/useractivity/\{id\}}

\paragraph{Returneaza toate activitatile utilizatorului, in sistem paginat}
\begin{itemize}
\item \textbf{timestamp}
 data si ora activitatii
\item \textbf{type}
 tipul activitatii (login, logout, download, vizionare date, etc.)
\item \textbf{details}
 in functie de activitate este posibil sa existe detalii, de exemplu la autentificare se vor inregistra IP-ul de pe care a fost accesat site-ul, tipul de navigator, sistemul de operare
\item \textbf{error}
 true sau false daca activitatea respectiva a produs o eroare
\item \textbf{errormessage}
 mesajul de eroare
\item \textbf{errordetails}
 detalii ale erorii, daca exista
 \end{itemize}
 \subsection*{GET: /admin/adminlist}

\paragraph{Returneaza lista tuturor administratorilor sistemului. O colectie de structuri de tip administrator, fiecare dintre ele continand urmatoarele:}
\begin{itemize}
\item \textbf{id}
 id-ul administratorului respectiv
\item \textbf{username}
 numele administratorului respectiv
\item \textbf{firstname}
 prenumele administratorului respectiv
\item \textbf{lastname}
 numele administratorului respectiv
\item \textbf{email}
 adresa de email a administratorului
\item \textbf{enabled}
 true daca administratorului este activ, false daca este blocat. 
\item \textbf{lastlogin}
 ultima autentificare a administratorului
 \end{itemize}
 \subsection*{GET: /admin/admininfo}

\paragraph{Returneaza datele unui singur administrator, impreuna cu o serie de colectii adiacente (activitate, setari).}
\begin{itemize}
\item \textbf{id}
 id-ul administratorului respectiv
\item \textbf{username}
 numele administratorului respectiv
\item \textbf{email}
 adresa de email a administratorului
\item \textbf{enabled}
 true daca administratorulu este activ, false daca este blocat. 
\item \textbf{firstname}
 prenumele
\item \textbf{lastname}
 numele
\item \textbf{middlename}
 alt prenume 
\item \textbf{image}
 fotografia
\item \textbf{phone}
 numarul de telefon al administratorului
\item\textbf{lastmessages}
 ultimele 10 mesaje trimise sau primite de administrator pe site-ul RODA \begin{itemize}
\item \textbf{timestamp}
 data si ora mesajului
\item \textbf{subject}
 subiectul mesajului
\item \textbf{direction}
 trimis sau primit
\item \textbf{read}
 true daca mesajul a fost citit, false daca nu
\item \textbf{message}
 corpul mesajului
 \end{itemize}
\item\textbf{activity}
 ultimele 50 de operatii pe care administratorul le-a executat pe sitepul RODA \begin{itemize}
\item \textbf{timestamp}
 data si ora activitatii
\item \textbf{type}
 tipul activitatii (login, logout, download, vizionare date, etc.)
\item \textbf{details}
 in functie de activitate este posibil sa existe detalii, de exemplu la autentificare se vor inregistra IP-ul de pe care a fost accesat site-ul, tipul de navigator, sistemul de operare
\item \textbf{error}
 true sau false daca activitatea respectiva a produs o eroare
\item \textbf{errormessage}
 mesajul de eroare
\item \textbf{errordetails}
 detalii ale erorii, daca exista
 \end{itemize}
 \end{itemize}
 \subsection*{GET: /admin/adminmessages/\{id\}}

\paragraph{Returneaza toate mesajele trimise sau primite de administrator, in sistem paginat.}
\begin{itemize}
\item \textbf{timestamp}
 data si ora mesajului
\item \textbf{subject}
 subiectul mesajului
\item \textbf{direction}
 trimis sau primit
\item \textbf{read}
 true daca mesajul a fost citit, false daca nu
\item \textbf{message}
 corpul mesajului
 \end{itemize}
 \subsection*{GET: /admin/adminactivity/\{id\}}

\paragraph{Returneaza toate activitatile administratorului, in sistem paginat}
\begin{itemize}
\item \textbf{timestamp}
 Data si ora activitatii
\item \textbf{type}
 Tipul activitatii (login, logout, download, vizionare date, etc.)
\item \textbf{details}
 In functie de activitate este posibil sa existe detalii, de exemplu la autentificare se vor inregistra IP-ul de pe care a fost accesat site-ul, tipul de navigator, sistemul de operare
\item \textbf{error}
 True sau false daca activitatea respectiva a produs o eroare
\item \textbf{errormessage}
 Mesajul de eroare
\item \textbf{errordetails}
 Detalii ale erorii, daca exista
 \end{itemize}
 \subsection*{GET: /admin/adminrights/\{id\}}

\paragraph{Returneaza toate permisiunile administratorului curent.}
\begin{itemize}
\item \textbf{rgroup}
 Categoria de drepturi 
\item \textbf{raction}
 Actiunea propriu-zisa, descrisa printr-un token de maximum 10 caractere fara spatii
\item \textbf{description}
 Descrierea actiunii
\item \textbf{rstatus}
 Permisiunile pe actiunea curenta
 \end{itemize}
 \subsection*{GET: /admin/admincheckperm/\{id\}}

\paragraph{Returneaza starea unei anumite permisiuni acordate sau nu administratorului}
\paragraph{Returneaza}
\begin{itemize}
\item \textbf{rgroup}
 Categoria de drepturi 
\item \textbf{raction}
 Actiunea propriu-zisa
\item \textbf{description}
 Descrierea actiunii
\item \textbf{raccess}
 True sau false in functie de stare
 \end{itemize}
 \subsection*{POST: /admin/usersave/\{id\}}

\paragraph{Parametri:}
\begin{itemize}
\item \textbf{id}
 Id-ul utilizatorului respectiv
\item \textbf{username}
 Username-ul utilizatorului respectiv
\item \textbf{email}
 Adresa de email a utilizatorului
\item \textbf{enabled}
 True daca utilizatorul este activ, false daca este blocat. 
 \end{itemize}
 \subsection*{POST: /admin/groupsave/\{id\}}

\paragraph{Parametri:}
\begin{itemize}
\item \textbf{name}
 Numele noului grup
\item \textbf{email}
 Adresa de email a utilizatorului
\item \textbf{description}
 Descrierea grupului respectiv
 \end{itemize}
 \subsection*{POST: /admin/addusertogroup/}

\paragraph{Parametri:}
\begin{itemize}
\item \textbf{groupid}
 Id-ul grupului 
\item \textbf{userid}
 Id-ul utilizatorului
 \end{itemize}
 \subsection*{POST: /admin/deleteuserfromgroup/}

\paragraph{Parametri:}
\begin{itemize}
\item \textbf{groupid}
 Id-ul grupului 
\item \textbf{userid}
 Id-ul utilizatorului
 \end{itemize}
 \subsection*{POST: /admin/deactivateuser/\{id\}}

\paragraph{Parametri:}
\begin{itemize}
\item \textbf{userid}
 Id-ul utilizatorului
 \end{itemize}
 \subsection*{POST: /admin/activateuser/\{id\}}

\paragraph{Parametri:}
\begin{itemize}
\item \textbf{userid}
 Id-ul utilizatorului
 \end{itemize}
 \subsection*{POST: /admin/dropuser/\{id\}}

\paragraph{Parametri:}
\begin{itemize}
\item \textbf{userid}
 Id-ul utilizatorului
 \end{itemize}
 \subsection*{POST: /admin/changeuserpassword/\{id\}}

\paragraph{Parametri:}
\begin{itemize}
\item \textbf{userid}
 Id-ul utilizatorului
\item \textbf{password}
 Parola utilizatorului
\item \textbf{controlpassword}
 Parola de control, trebuie sa fie identica cu cealalta parola
 \end{itemize}
 \subsection*{POST: /admin/messageuser/\{id\}}

\paragraph{Parametri:}
\begin{itemize}
\item \textbf{userid}
 Id-ul utilizatorului
\item \textbf{subject}
 Subiectul mesajului 
\item \textbf{message}
 Corpul mesajului
 \end{itemize}
 \subsection*{POST: /admin/messagegroup/\{id\}}

\paragraph{Parametri:}
\begin{itemize}
\item \textbf{groupid}
 Id-ul grupului
\item \textbf{subject}
 Subiectul mesajului 
\item \textbf{message}
 Corpul mesajului
 \end{itemize}
 \subsection*{POST: /admin/adminsave/\{id\}}

\paragraph{Modifica sau introduce in sistem un nou administrator. Daca se specifica parametrul id, inregistrarea existenta este modificata conform apelului, daca nu, se va crea o noua inregistrare}
\paragraph{Parametri:}
\begin{itemize}
\item \textbf{id}
 Id-ul administratorului respectiv
\item \textbf{username}
 Username-ul administratorului respectiv
\item \textbf{email}
 Adresa de email a administratorului
\item \textbf{enabled}
 True daca administratorulu este activ, false daca este blocat. 
\item \textbf{firstname}
 Prenumele
\item \textbf{lastname}
 Numele
\item \textbf{middlename}
 Alt prenume 
\item \textbf{image}
 Fotografia
\item \textbf{phone}
 Numarul de telefon al administratorului
 \end{itemize}
 \subsection*{POST: /admin/deactivateuser/\{id\}}

\paragraph{Parametri:}
\begin{itemize}
\item \textbf{adminid}
 Id-ul administratorului
 \end{itemize}
 \subsection*{POST: /admin/activateuser/\{id\}}

\paragraph{Parametri:}
\begin{itemize}
\item \textbf{admin}
 Id-ul utilizatorului
 \end{itemize}
 \subsection*{POST: /admin/changeadminpassword/\{id\}}

\paragraph{Parametri:}
\begin{itemize}
\item \textbf{adminid}
 Id-ul administratorului
\item \textbf{password}
 Parola administratorului
\item \textbf{controlpassword}
 Parola de control, trebuie sa fie identica cu cealalta parola
 \end{itemize}
 \subsection*{POST: /admin/addadminpermissions/\{id\}}

\paragraph{Adauga o anumita permisiune administratorului. }
 \subsection*{POST: /admin/dropadminpermissions/\{id\}}

\paragraph{Revoca o anumita permisiune administratorului. }
