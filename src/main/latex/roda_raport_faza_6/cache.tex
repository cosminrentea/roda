\subsection{Sistemul de Cache}

EhCache (\url{http://www.ehcache.org}) este un software utilizat pentru caching. 
Este o librarie Java care este distribuita atat sub o licenta open-source (pentru functionalitatile uzuale) cat si sub o licenta proprietara (pentru functionalitatile avansate).

Ehcache consta dintr-un CacheManager, care gestioneaza mai multe Cache-uri. 
Un Cache contine Elemente, care sunt in esenta perechi de tip nume-valoare. 
Cache-urile sunt implementate fizic, in memorie si/sau pe disc.

Ehcache poate fi folosit in urmatoarele topologii:
\begin{description}
\item 
[Standalone] 
--  
Setul de date din cache este tinut in  nodul pe care ruleaza aplicatia.
Orice alte noduri sunt independente , cu nici o comunicare intre ele.
Daca caching-ul independent este utilizat, în cazul în care exista mai multe noduri de aplicatii care ruleaza aceeasi aplicatie, atunci există între ele "weak-consistency".
Ele contin valori consistente de date imuabile sau dupa "timpul de a trai" al unui element a expirat si elementul trebuie sa fie reincarcat. 
Nodurile independente pot fi rulate folosind Ehcache open-source.

\item
[Distributed]
--
Datele sunt tinute într-un Array Terracotta Server, cu un subset de date utilizate recent retinut in fiecare nod cache aplicatie. 
Topologia distribuita, disponibila cu BigMemory Max, sustine un set foarte bogat de moduri de consistenta.

\item
[Replicated]
-–
Setul de date din cache este tinut in fiecare nod aplicatie si date sunt copiate sau invalidate in cluster fara locking.
Replicarea poate fi fie asincrona sau sincrona, unde thread-urile de scriere se blocheaza cat timp se intampla propagarea.
Singurul mod consistenta disponibil în aceasta topologie este "weak-consistency". Nodurile pot fi rulate cu Ehcache open-source.
\end{description}

In framework-ul Spring, EhCache este cea mai complexa alternativa ce poate fi folosita in layer-ul de "cache abstraction" \footnote{\url{http://docs.spring.io/spring/docs/3.2.x/spring-framework-reference/html/cache.html}}.
Se utilizeaza adnotari pentru acele metode ale caror procesari/rezultate pot fi cache-uite. 
SpEL (Spring Expression language) poate fi folosit in interiorul adnotarilor pentru definirea cheilor sau a altor parametri ai cache-urilor.

Aplicatia web a RODA foloseste pentru moment 2 instante de cache-uri: pentru paginile generate de CMS si pentru thumbnail-uri.

Astfel, este permisa stocarea unei pagini web generate de sistemul CMS (vezi sectiunea \ref{generarea_paginilor}) 
pe o perioada determinata de timp, evitandu-se generarea acesteia la fiecare solicitare din intervalul respectiv. 
Ulterior stergerii din cache a unei pagini, in momentul urmatoarei cereri de accesare aceasta este regenerata si introdusa in cache-ul dedicat.

Un alt cache este definit pentru stocarea imaginilor de tip \emph{"thumbnails"} (vezi sectiunea \ref{thumbnails}), 
indiferent cum sunt acestea procesate/obtinute (dimensiuni, algoritm).
Intrucat metode diferite genereaza thumbnail-uri diferite, cheia unica definita pentru acest cache include si numele metodei pe langa numele fisierului si dimensiunile sale:
\begin{lstlisting}[breaklines=true]
@Cacheable(value = "thumbnails", key = "{#root.methodName, #url, #alias, #height, #width}")
\end{lstlisting}

Cele 2 cache-uri descrise mai sus sunt configurate in fisierul "ehcache.xml" cu parametri precum timpul de viata / expirare, dimensiunea cache-ului (in memorie, pe disk), strategia de caching (de ex. LRU), strategia de overflow etc.

EhCache (manager-ul de cache-uri) este incarcat in aplicatia Spring in ca un Bean in fisierul de configurare "applicationContext.xml".