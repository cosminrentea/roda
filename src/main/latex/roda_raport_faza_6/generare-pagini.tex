\subsection{Generarea paginilor}
\label{generarea_paginilor}

Controller-ul care primeste solicitarea pentru pagina, prin intermediul unei cereri prefixate de "/page", apeleaza metoda \textbf{generatePage(url)} a serviciului \emph{RodaPageService}. 
Portiunea de URL care urmeaza lui "/page" va specifica, de fapt, o succesiune de fragmente de URL corespunzatoare unei ierarhii de pagini din baza de date. 
Pentru o pagina din baza de date, campul "url" contine, de fapt, doar fragmentul final al URL-ului paginii. 
Identificarea URL-ului integral (unic) al paginii se realizeaza prin parcurgerea bottom-up a ierarhiei, pana la pagina radacina de pe drumul respectiv. 
De regula, pagina radacina va fi "ro" sau "en", corespunzator limbii paginilor din arborescenta de sub acestea. 
De exemplu, \emph{<server>:8080/roda/page/ro/acasa} va genera si returna pagina al carei camp URL din baza de date are valoarea "acasa", si este subordonata paginii "ro". 

Metoda \textbf{generatePage(URL)} include efectuarea urmatoarelor actiuni:

\begin{enumerate}
\item{In functie de parametrul primit (URL), cauta inregistrarea corespunzatoare din tabelul de pagini al bazei de date.}
\item{Cu informatia despre pagina curenta, apeleaza metoda getLayout(cmsPage, url). Aceasta:}
\begin{itemize}
\item{regaseste continutul layout-ului in baza de date}
\item{inlocuieste toate codurile din continutul layout-ului, cu exceptia lui \emph{[[Code: PageContent]]}}
\item{returneaza continutul layout-ului.}
\end {itemize}
\item{Pe baza paginii curente si a layout-ului obtinut anterior, este apelata metoda replacePageContent(layoutContent, cmsPage). Aceasta determina daca exista codul \emph{[[Code: PageContent]]} in continutul layout-ului, iar in caz afirmativ realizeaza urmatoarele actiuni:}
\begin{itemize}
\item {regaseste continutul paginii in baza de date}
\item{inlocuieste toate codurile din continutul paginii}
\item {inlocuieste codul \emph{[[Code: PageContent]]} cu continutul paginii, astfel prelucrat.}
\end{itemize}
\end{enumerate}

\bigskip

Pagina astfel generata este trimisa, ca atribut al modelului, unui fisier show.jsp; 
in acest fisier, codul generat al paginii va corespunde sectiunii /emph{body}.
Calea (referinta) catre fisierul show.jsp este returnata de catre metoda de afisare din controller-ul de pagina, \emph{RodaPageController}.

O portiune a codului sursa al acestei metode este urmatorul:

\begin{lstlisting}[breaklines=true]
@RequestMapping(value = "/**", produces = "text/html")
public String show(HttpServletRequest request, Model uiModel) {

	String url = (String) request.getAttribute(HandlerMapping.PATH_WITHIN_HANDLER_MAPPING_ATTRIBUTE);
	..........
	String pageBody = rodaPageService.generatePage(url)[0];
	..........

	uiModel.addAttribute("pageBody", pageBody);

	return "rodapage/show";
}
\end{lstlisting}	