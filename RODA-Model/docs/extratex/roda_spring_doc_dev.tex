La specificarea in instructiunile de mai jos a sistemului de operare:
\begin{description} 
\item [MacOSX =] OS X version 10.8.3, 64 bit
\item [Ubuntu =] Ubuntu 12.04LTS, kernel 3.2.0-39-generic-pae, 32bit
\item [Postgresql =] Postgresql 9.2
\item [DbWrench =] DbWrench 2.3.2
\item [STS =] Spring Tool Suite 3.2.0.RELEASE - based on Eclipse Juno 3.8.2
\item [Maven =] Maven 3
\end{description}

\section{Instalare si configurare - servicii si software aditional}

Se instaleaza Postgresql ca server pe statia locala (preferabil, pornit automat ca serviciu la bootare):
\begin{description}
\item[Ubuntu:] In linia de comanda:
\begin{lstlisting}
	sudo apt-get install postgresql
\end{lstlisting}

[Optional] Se configureaza in fisierele de configurare ale Postgresql drepturile
de acces ale clientilor la server (???).

\item [MacOSX:] 
Se instaleaza si ruleaza Postgres.app (descarcat de la http://postgresapp.com ).
Se bifeaza in meniu optiunea "Automatically Start at Login".
\end{description}

Se instaleaza pachete necesare:
\begin{description}
\item[Ubuntu:]
Se ruleaza urmatoarele comenzi, din shell:
\begin{lstlisting}[breaklines=true]
sudo apt-get install openjdk-6-jdk maven perl r-base r-base-dev pgadmin3 libtest-www-selenium-perl graphviz rubygems
sudo gem install vmc
\end{lstlisting}
\item[MacOSX:]
Se instaleaza anterior MacPorts si dependentele necesare (instructiuni la: http://www.macports.org/install.php ).

Se ruleaza urmatoarele comenzi, din shell:
\begin{lstlisting}[breaklines=true]
sudo port install maven3 perl5 R pgadmin3 p5.12-test-www-selenium graphviz gem
sudo gem install vmc
\end{lstlisting}
\end{description}

Se verifica instalarea Java si tipul de JVM disponibil (32 sau 64 bit), cu comanda:
\begin{lstlisting}
	java -version
\end{lstlisting}

Se creeaza in directorul \$HOME al utilizatorului un fisier numit ".mavenrc"
(ce contine setarile generale pentru Maven, in acest caz legate de memoria utilizata);
in fisier se scrie:
\begin{lstlisting}
	MAVEN_OPTS="-Xmx1024m -XX:MaxPermSize=256m"
	export MAVEN_OPTS
\end{lstlisting}

Se instaleaza rJava in R, cu urmatoarele comenzi in shell si apoi in shell-ul R
(cu alegerea in timpul instalarii a mirror-ului de folosit, dintr-o fereastra):
\begin{lstlisting}
	sudo R CMD javareconf
	R
	install.packages('rJava')
\end{lstlisting}

Se instaleaza DbWrench de la \url{http://www.dbwrench.com}.

\section{Instalarea si configurarea IDE-ului si a proiectului}

\subsection{Instalare Spring Tool Suite (STS)}
Descarcati STS de pe pagina:
\url{http://www.springsource.org/downloads/sts-ggts} .

Alegeti pentru descarcarea versiunea STS bazata pe Eclipse 3.8.2; pe website,
sectiunea respectiva este intitulata:
"Milestone Version - Spring Tool Suite 3.2.0.RELEASE - based on Eclipse Juno 3.8.2"

Instalati STS, eventual din linia de comanda, cu optiunile default: 
Next -> ... -> Finish.

\subsection{Configurare STS (inainte de pornire)}
In fisierul "STS.ini" (in Windows/Linux, acesta este in folder-ul radacina al STS; in MacOS X, in folder-ul "STS.app/Contents/MacOS/"),
se modifica memoria maxima utilizabila de STS.

In "STS.ini", linia care incepe cu \emph{-Xmx} (memoria maxima utilizabila de
STS) trebuie modificata astfel:
\begin{itemize} 
\item daca JVM este pe 64 biti:

	-Xmx3072m
\item daca JVM este pe 32 biti:

	-Xmx1536m
\end{itemize}

\subsection{Instalare Extensii si Plugin-uri in STS}

Porniti STS, eventual din linia de comanda, in folder-ul radacina al instalarii STS:
\begin{lstlisting}	
	./STS &
\end{lstlisting}

Din Dashboard (meniu Help -> Dashboard), click pe tab-ul "Extensions"
(pozitionat jos, in Dashboard), bifati in lista si apoi instalati (butonul
"Install"):
\begin{itemize}
  \item Cloud Foundry Eclipse Integration
  \item Edgewall Trac
\end{itemize} 

Restartati STS.

Instalati in STS ultima versiune de Subclipse (cu suport pentru Subversion 1.7), prin meniu Help -> Install New Software:

URL = \url{http://subclipse.tigris.org/update_1.8.x}
	
Selectati si instalati de la acest update-site: "Subclipse" si "SVNKit".

Restartati STS.

Configurati in STS Subclipse pentru a utiliza SVNKIT: 
meniu Window -> Preferences -> Team -> SVN, 
apoi la "SVN Interface" (mai jos in pagina de optiuni aparuta) se alege Client = "SVNKit (pure Java) SVNKit v1.7...."

Instalati in STS plugin-ul IDE pentru Perl, prin meniu Help -> Install
New Software:
Perl EPIC, URL = \url{http://e-p-i-c.sf.net/updates/testing}

\subsection{Configurare Task Repository}
In STS, in view-ul "Task List", butonul "Add Repository", de tipul "Trac", 
iar in pasul urmator:

Server URL:    \url{http://fisiere.dyndns.org:8888/roda}

Debifare optiune "Anonymous"

Completare User si Parola (specifice)

Bifare "Save password"

"Validate Settings"

"Finish"

La ultimul pas, se adauga un Query legat de acest Repository (de exemplu, se completeaza doar un nume de tip "All" si se apasa Finish).

\subsection{Preluare proiecte din SVN repository}
In STS, adaugati SVN repository (in perspectiva "SVN Repository
Exploring"), 

URL = \url{svn://fisiere.dyndns.org/roda}

Faceti \emph{Checkout} din SVN repository (click-dreapta pe numele
proiectului, apoi 'Checkout') pentru urmatoarele proiecte:
\begin{description}
\item [roda:] aplicatia web Java / Spring / Roo
\item [dbwrench2-files:] fisierele dbWrench, SQL si scripturile asociate
\item [RODA-Model:] modelul Perl si documentatia
\end{description}

In STS, se face upgrade la Subversion 1.7 (pt. working copy) pentru fiecare din proiectele de mai sus: 
click-dreapta pe proiect in view-ul "Package Explorer" sau "Navigator", si apoi in meniul aparut:
Team -> Upgrade -> 'OK'.

\section{Configurarea locala a proiectului}

\subsection{Compilarea pentru prima data a proiectului}

In linia de comanda, in folderul-radacina al proiectului "roda":
\begin{lstlisting}
mvn -e -X clean package
\end{lstlisting}

Toate pachetele necesare sunt descarcate cand se face prima compilare (care poate dura mai mult).

In STS, click-dreapta in view-ul "Package Explorer" pe numele proiectului ("roda"),
apoi din meniul aparut:
Maven -> Update Project -> 'OK'.

STS va deschide un view numit "Roo shell" unde se pot da comenzi Roo referitoare
la proiect.

\subsection{Configurarea Serverului Local (in STS)}
Se trage (drag-and-drop) proiectul "roda" din view-ul "Package Explorer" 
peste serverul local prezent in view-ul "Servers" (VMware vFabric tc Server ...).

\subsection{Configurarea serverului Postgresql local}
Folosind pentru conectare in linia de comanda
\begin{lstlisting}
psql -h localhost
\end{lstlisting}
se dau comenzile:
\begin{lstlisting}
CREATE USER roda;
CREATE DATABASE roda OWNER roda ENCODING 'UTF8';
\end{lstlisting}

\subsection{Fisierele de configurare ale proiectului}
\begin{description}
\item[Conexiunea la baza de date trebuie configurata in:]
src/main/resources/META-INF/spring/database.properties
\item[Hibernate si JPA] (de ex. modul de lucru: update / create / create-drop /
validate):
src/main/resources/META-INF/persistence.xml
\item[Logging-ul:]
src/main/resources/log4j.properties
\item[Spring (Beans, properties etc.):]
src/main/resources/META-INF/spring/applicationContext.xml
\item[Spring Security Authentication:]
src/main/resources/META-INF/spring/applicationContext-security.xml
\item[Spring Security ACL:]
src/main/resources/META-INF/spring/applicationContext-acl.xml
\item[Conexiunea cu serverul Solr:]
src/main/resources/META-INF/spring/solr.properties
\item[Configurarea aplicatiei web RODA:]
src/main/resources/META-INF/spring/roda.properties
\item[Procedurile de build (Maven):]
/pom.xml
(aici se pot schimba si parametri: portul folosit de Tomcat la rularea aplicatiei; rularea sau nu a testelor etc.)
\end{description}

\section{Rularea locala a proiectului}

\subsection{Rularea din STS, pe Server Local}
Se (re-)porneste serverul folosind butoanele din view-ul "Servers".

Daca portul local 8080 era liber (se poate edita dupa dublu-click pe server,
aplicatia e disponibila la:

\url{http://localhost:8080/roda}

\subsection{Rularea din linia de comanda, pe un Server Local ad-hoc (tip
Tomcat)}
Din linia de comanda, in folder-ul radacina al proiectului "roda", cu comanda:
\begin{lstlisting}
mvn tomcat:run
\end{lstlisting}
Daca portul local 8080 era liber (portul se poate schimba in "pom.xml"),
aplicatia e disponibila la:

\url{http://localhost:8080/roda}

\subsection{Rularea automata a testelor web (Selenium)}
Din linia de comanda, in folder-ul radacina al proiectului "roda", cu comanda:
\begin{lstlisting}
mvn tomcat:run selenium:selenese
\end{lstlisting}

\section{Rularea proiectului din STS, pe CloudFoundry (NERECOMANDAT)}

!!! ATENTIE !!!
Accesul la acest cont/server este shared (e folosit inclusiv de catre Jenkins), 
pot aparea suprapuneri nedorite intre variante diferite / dezvoltatori diferiti !!!

Se adauga in view-ul "Servers" un nou server remote: click-dreapta -> 'New'
-> 'Server' -> VMware - Cloud Foundry -> 'Next'.

Se completeaza wizard-ul conform urmatorilor pasi:
\begin{itemize}
  \item 
Email: roda.devel@gmail.com
  \item 
Parola: RodaAdor
  \item 
"Validate Account"
  \item 
"Next"
  \item 
Se muta proiectul "roda" din lista "Available" in lista "Configured"
  \item 
"Next"
  \item 
In fereastra "Application details", se selecteaza "Application Type" -> "Spring"
  \item 
"Next"
  \item 
Deployed URL: roda.cloudfoundry.com
  \item 
Memory Reservation: 2048 M
  \item 
"Next"
  \item 
Se bifeaza serviciul "roda-postgres" (baza de date Postgres) din lista aparuta.
  \item 
"Finish"
\end{itemize}

Proiectul este disponibil online permanent la adresa:

\url{http://roda.cloudfoundry.com}

\section{Instalarea si configurarea Jenkins}

???
