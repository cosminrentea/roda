
\medskip

Pentru autentificare este utilizata si adaptata solutia propusa de Spring Security 3.2.x (\url{http://docs.spring.io/spring-security/}).

Fisierul de configurare in contextul Spring este unul separat: 'applicationContext-security.xml'. 

\medskip

Autentificarea se face conform unui \c{s}ablon de tipul Users -- Roles.

Utilizatorii precum si hash-urile parolelor lor (SHA-256) sunt stocate intr-un tabel specific din baza de date ('users').
Acest tabel este folosit de implementari standard ale Spring Security; 
managerul de autentificare foloseste un provider de tip JDBC ('UsersDetailsService'), 
ce este instantiat prin tag-ul 'jdbc-user-service' din fisierul de configurare. 

Din tabelul 'users' al bazei de date nu se sterg utilizatori (tabelul fiind referit de numeroase alte campuri din celelalte tabele ale bazei de date); 
acesti utilizatori pot fi insa oricand dezactivati sau reactivati.
Tabelul 'users' este relationat cu alte tabele ce corespunzand modelului, 
de exemplu atunci cand este relevant sa se cunoasca cine a executat o anumita operatiune (import de studii, completare metadate, rectificare etc.).

Tabelul 'authorities' listeaza toate rolurile care sunt permise unui anumit utilizator (de ex. administrator, operator date, operator backup, cercetator, vizitator etc.); 
lista de roluri este predefinita, cf. raportului fazei I.

\medskip

Aplicatia web are definite interceptoare de URL-uri care permit accesul la anumite zone doar daca utilizatorul este autentificat si are un anumit rol.

Interfata aplicatiei permite utilizatorilor cu rol de administrator sa creeze noi utilizatori, sa ii activeze/dezactiveze si sa le acorde acestora unul sau mai multe roluri in cadrul aplicatiei.

Accesul la componenta DataBrowser a aplicatiei se face momentan doar dupa logarea ca utilizator administrator.

De asemenea, meniul aplicatiei se schimba in functie de rolurile utilizatorului logat.

\medskip

Spring Security are un design modular si permite extinderea modelului de securitate prin autentificarea cu LDAP, OpenID sau Shibboleth (SAML2).
Aceasta solutie de autentificare permite ca in continuare pentru autorizare sa fie utilizata si adaptata solutia Spring Security ACL --
se pot utiliza unele implementari standard din Spring Security (Access Control Lists ierarhice aplicate asupra diferitelor clase din model si servicii). 
Serviciile ce permit accesul la model sunt adnotate folosind SpEL
(Spring Expression Language).
Pentru a mari viteza de decizie este utilizata solutia de caching Ehcache (\url{http://ehcache.org/}). 
Se face auditarea simpla a accesului (incercare, succes / esec). 
