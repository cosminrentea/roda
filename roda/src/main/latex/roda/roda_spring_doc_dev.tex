La specificarea in instructiunile de mai jos a sistemului de operare:
\begin{description} 
\item [MacOSX =] OS X version 10.8.4, 64 bit
\item [Ubuntu =] Ubuntu 12.04LTS, kernel 3.2.0-39-generic-pae, 32bit
\item [OpenSUSE = ] OpenSUSE 12.2
\item [Postgresql =] Postgresql 9.2.4
\item [Hibernate =] Hibernate 4.1.8
\item [DbWrench =] DbWrench 2.3.2
\item [STS =] Spring Tool Suite 3.2.0.RELEASE - based on Eclipse Juno 3.8.2
\item [Maven =] Maven 3
\item [JVM =] Java Virtual Machine (JVM) 1.6
\item [Jenkins =] Jenkins 1.517
\item [Solr =] Solr 4.3
\item [Sencha Architect =] Sencha Architect 2.2.2
\end{description}

\section{Instalare si configurare - servicii si software aditional}

\subsection{Instalarea pachetelor necesare}
\begin{description}
\item[Ubuntu:]
Se ruleaza urmatoarele comenzi, din shell:
\begin{lstlisting}[breaklines=true]
sudo apt-get install openjdk-6-jdk maven perl r-base r-base-dev pgadmin3 libtest-www-selenium-perl htmldoc graphviz rubygems postgresql-server firefox solr

sudo gem install vmc
\end{lstlisting}

\begin {comment}
\item[OpenSUSE:]
% TODO ???
Maven se instaleaza manual sau dintr-un repository non-standard.

R se instaleaza manual sau dintr-un repository non-standard.

Test::WWW::Selenium se instaleaza din CPAN.

Se ruleaza urmatoarele comenzi, din shell:
\begin{lstlisting}[breaklines=true]
sudo zypper in openjdk-6-jdk perl pgadmin3 htmldoc graphviz ruby ruby-devel rubygems postgresql

sudo gem install vmc
\end{lstlisting}
\end{comment}

\item[MacOSX:]

Se instaleaza MacPorts si toate dependentele necesare (instructiuni complete la:
\url{http://www.macports.org/install.php} ).

Se ruleaza urmatoarele comenzi, din shell:
\begin{lstlisting}[breaklines=true]
sudo port install maven3 perl5 R pgadmin3 p5.12-test-www-selenium htmldoc graphviz rb-rubygems postgresql92 firefox-x11 apache-solr4
	
sudo gem install vmc
\end{lstlisting}
\end{description}

\subsection{LaTeX}
O distributie LaTeX este necesara pentru generarea documentatiei:

\begin{description}
\item[Ubuntu:] 
\begin{lstlisting}[breaklines=true]
	sudo apt-get install texlive
\end{lstlisting}
\item[MacOSX:] 
MacTeX, instalata de la \url{http://tug.org/mactex/} 
\end{description}

Optional, se pot instala editoare de LaTeX - de ex., in MacOSX  se poate utiliza TeXShop (\url{http://pages.uoregon.edu/koch/texshop/}).

\subsection{Maven si Java Virtual Machine (JVM)}
Se creeaza in directorul \emph{\$HOME} al utilizatorului un fisier numit
'.mavenrc' 
(ce contine setarile generale pentru Maven, in acest caz legate de memoria utilizata);
in fisier se scrie:
\begin{lstlisting}[breaklines=true]
	MAVEN_OPTS='-Xmx1024m -XX:MaxPermSize=256m'
	export MAVEN_OPTS
\end{lstlisting}

\label{java_version}
Se verifica instalarea Java (e necesar minim Java 1.6) si tipul de JVM
utilizat (32 sau 64 bit), in shell:
\begin{lstlisting}
	java -version
\end{lstlisting}

\subsection{rJava}
Se instaleaza rJava in R, cu urmatoarele comenzi in shell si apoi in shell-ul R
(se alege in timpul instalarii site-mirror-ul folosit, dintr-o fereastra;
se iese la final din shell-ul R cu Ctrl+D):
\begin{lstlisting}
	sudo R CMD javareconf
	R
	install.packages('rJava')
\end{lstlisting}

\subsection{DbWrench}
Se instaleaza DbWrench de pe site-ul producatorului:
\url{http://www.dbwrench.com}.

\section{Instalarea si configurarea IDE-ului si a proiectului}

\subsection{Instalare Spring Tool Suite (STS)}
Descarcati STS de pe pagina:
\url{http://www.springsource.org/downloads/sts-ggts},
facand click direct pe link-ul \emph{'Just take me to the download page'}.

Alegeti pentru descarcarea versiunea STS bazata pe Eclipse 3.8.2, extensia
fisierului de salvat fiind '.sh', iar arhitectura cea stabilita de versiunea
JVM (32 sau 64 bit, vezi \ref{java_version}); pe website, sectiunea respectiva
este cea intitulata:
\begin{lstlisting}[breaklines=true]
Milestone Version - Spring Tool Suite 3.2.0.RELEASE - based on Eclipse Juno 3.8.2
\end{lstlisting}

Instalati STS din shell (fara 'sudo'), cu optiunile default (path-ul default este \emph{\$HOME/springsource} ):
'Next' -> ... -> 'Finish'.

\subsection{Configurarea STS inainte de rulare}
In fisierul \emph{STS.ini} (in Ubuntu/OpenSUSE, acesta este in folder-ul
radacina al STS, ales la pasul anterior; in MacOS X, este in sub-folder-ul
'STS.app/Contents/MacOS/'), se modifica memoria maxima utilizabila de STS.

In \emph{STS.ini}, linia care incepe cu \emph{-Xmx} 
(memoria maxima utilizabila de STS) 
trebuie modificata astfel:
\begin{itemize} 
\item daca JVM este pe 64 biti:
\begin{lstlisting}
	-Xmx3072m
\end{lstlisting}
\item daca JVM este pe 32 biti:
\begin{lstlisting}
	-Xmx1536m
\end{lstlisting}
\end{itemize}

\subsection{Configurarea plugin-urilor si extensiilor STS}

Porniti STS, eventual din linia de comanda, in folder-ul radacina al instalarii STS:
\begin{lstlisting}	
	./STS &
\end{lstlisting}

Din Dashboard (meniu Help -> Dashboard), click pe tab-ul 'Extensions'
(pozitionat stanga-jos, in Dashboard), bifati in lista si apoi instalati
(butonul 'Install', dreapta-jos) urmatoarele extensii:
\begin{itemize}
\item
Cloud Foundry Eclipse Integration
\item
Edgewall Trac
\end{itemize} 

Restartati STS.

Instalati in STS ultima versiune de Subclipse (cu suport pentru Subversion 1.7),
prin meniu 'Help' -> 'Install New Software':
\begin{itemize}
\item 
nume = Subclipse 1.8
\item
URL = \url{http://subclipse.tigris.org/update_1.8.x}
\end{itemize}

Selectati si instalati de la acest update-site ambele seturi de plugin-uri:
'Subclipse' si 'SVNKit'.

Restartati STS.

Configurati Subclipse pentru a utiliza SVNKIT:
\begin{enumerate}
\item 
click pe meniu 'Window' -> 'Preferences' -> alegere din lista 'Team' -> 'SVN'
(pe MacOSX: meniu 'Spring Tool Suite' -> 'Preferences' -> 'Team' -> 'SVN')
\item
la 'SVN Interface' (mai jos in pagina de optiuni aparuta) se alege Clientul
folosit:

'SVNKit (pure Java) SVNKit v1.7....'
\end{enumerate}

(Optional) Instalati plugin-ul IDE pentru Perl, prin meniu 'Help' -> 
'Install New Software':
\begin{itemize}
\item 
nume = Perl EPIC
\item
URL = \url{http://e-p-i-c.sf.net/updates/testing}
\end{itemize}

\subsection{Configurare Task Repository}
In STS, in view-ul 'Task List', butonul 'Add Repository', de tipul 'Trac', 
iar in pasul urmator:
\begin{itemize}
\item 
Server URL:    \url{http://fisiere.dyndns.org:8888/roda}
\item
Debifare optiune 'Anonymous'
\item
Completare User si Parola (specifice)
\item
Bifare 'Save password'
\item
'Validate Settings'
\item
'Finish'
\end{itemize}

La ultimul pas, se adauga un Query legat de acest Task-Repository (de exemplu,
se completeaza doar un nume - precum 'ALL' - si se apasa 'Finish').

\subsection{STS Preferences}
In fereastra 'Preferences' a STS:
\begin{itemize}
\item
in Java -> Editor -> Save Actions, se bifeaza: 'Perform the selected actions on save' ; ' Format source code' (format all lines).
\item
in Java -> Code Style -> Formatter, se selecteaza profilul 'Eclipse [built-in]', se apasa 'Edit', in tab-ul 'Line Wrapping' , campul 'Maximum line width' se completeaza cu '120' (in loc de '80').
Se completeaza in partea de sus numele profilului: 'RODA' , se apasa 'OK'. Se selecteaza
\item
(optional) in Java -> Appearance -> Members Sort Order, se bifeaza 'Sort member in same category by visibility'.
\item
(optional) in SpringSource -> Dashboard, se debifeaza 'Show Dashboard on Startup'.
\end{itemize}

\subsection{Preluare proiecte in STS din repository}
In STS, adaugati un SVN repository 
(in perspectiva 'SVN Repository Exploring'), 
folosind URL: \url{svn://fisiere.dyndns.org/roda} .

Faceti \emph{Checkout} succesiv din SVN repository 
(click-dreapta pe numele proiectului / directorului, apoi 'Checkout') 
pentru:
\begin{description}
\item [roda:] aplicatia web Java / Spring
\item [dbwrench2-files:] fisiere DbWrench (format XML + SQL), scripturi pentru operatii asupra bazei de date
\item [(optional) RODA-Model:] modelul Perl
\end{description}

\begin{comment}
In STS, se face upgrade la Subversion 1.7 (pt. working copy) pentru fiecare din proiectele de mai sus: 
in perspectiva 'Spring', 
in view-ul 'Package Explorer' sau in view-ul 'Navigator', 
click-dreapta pe numele proiectului respectiv,
si apoi in meniul aparut:
'Team' -> 'Upgrade' -> 'OK'. 
Poate aparea un mesaj de eroare/warning, care indica
faptul ca working-copy local este deja conform versiunii 1.7 a SVN.
\end{comment}

\section{Configurarea locala a proiectului}
\subsection{Compilarea pentru prima data a proiectului}

In linia de comanda, in folderul-radacina al proiectului 'roda':
\begin{lstlisting}
	mvn -e -X clean package
\end{lstlisting}
Toate pachetele necesare sunt descarcate atunci cand se face prima compilare - care poate dura mai mult.

In STS, se alege perspectiva 'Spring', 
apoi click-dreapta in view-ul 'Package
Explorer' pe numele proiectului ('roda'), 
apoi din meniul aparut:
'Maven' -> 'Update Project', 
apoi 'OK'.

% STS va deschide un view numit 'Roo shell' unde se pot da comenzi Roo referitoare la proiect.

\subsection{Fisierele de configurare ale aplicatiei}
Toate detaliile necesare se regasesc in sectiunea
dedicata fisierelor de configurare din capitolul dedicat componentelor/arhitecturii aplicatiei: 
\ref{fisiere_configurare}.

Configurarile \emph{default} din aceste fisiere nu trebuie neaparat
schimbate, daca se urmeaza intocmai acest ghid de instalare/configurare.

\subsection{Configurarea Serverului de Aplicatii local (in STS)}
Se trage (drag-and-drop) proiectul 'roda' din view-ul 'Package Explorer' 
peste serverul local prezent in view-ul 'Servers' (VMware vFabric tc Server ...).


\subsection{Instalarea si configurarea serverului Postgresql local}
\label{postgresql_configurare}

\begin{description}
\item[MacOSX:]

Se instaleaza si ruleaza \emph{Postgres.app} (descarcat de la \url{http://postgresapp.com} ).
Se bifeaza in meniul programului optiunea 'Automatically Start at Login'.

\item[orice alt sistem de operare:]

Se configureaza Postgresql ca server pe statia locala 
(preferabil, pornit automat ca serviciu la bootare). 
Se pot configura in fisierele de configurare ale Postgresql tipurile de autentificare ale clientilor la server 
(de ex. pentru a permite accesul prin
internet, nu doar de la statia locala / reteaua locala).
% TODO mai precis ???
\end{description}

In shell, din directorul 'dbwrench-files' (dupe check-out-up din SVN), 
se ruleaza urmatoarea comanda
(pentru crearea utilizatorului, a bazei de date si a schemelor auxiliare):
\begin{lstlisting}
	./create-roda-db.sh
\end{lstlisting}

\subsection{Configurarea serverului Solr local}
\label{solr_configurare}

In MacOSX, Solr se porneste din shell cu comanda:
\begin{lstlisting}[breaklines=true]
	sudo solr4
\end{lstlisting}
serverul fiind disponibil la URL: \url{http://localhost:8983/solr/}

Pentru configurarea Solr este folosita schema din subfolder-ul 'examples'. 
% TODO mai precis ???

\section{Rularea locala a proiectului}

De fiecare data, inainte de pornirea aplicatiei, trebuie sa fie deja pornite:
\begin{description}
\item[Postgresql] ca un server local pe portul 5432, utilizand baza de date creata (vezi \ref{postgresql_configurare})
\item[Solr] ca un server local pe portul 8983, folosind Jetty (vezi \ref{solr_configurare})
\end{description}

\subsection{Rularea din STS, pe Server Local}
Se (re-)porneste serverul de aplicatii local folosind butoanele din view-ul
'Servers'.

Daca portul local 8080 era liber (portul dorit se poate edita dupa dublu-click pe
server, in lista din view-ul 'Servers'), aplicatia e disponibila la:

\url{http://localhost:8080/roda}

\subsection{Rularea din linia de comanda, pe un Server Local ad-hoc de tip
Tomcat}
Din linia de comanda, in folder-ul radacina al proiectului 'roda', cu comanda:
\begin{lstlisting}
	mvn tomcat:run
\end{lstlisting}
Daca portul local 8080 era liber (portul se poate schimba in 'pom.xml'),
aplicatia e disponibila la:

\url{http://localhost:8080/roda}

\subsection{Rularea automata a testelor web (Selenium)}
Din linia de comanda, in folder-ul radacina al proiectului 'roda', cu comanda:
\begin{lstlisting}
	mvn tomcat:run selenium:selenese
\end{lstlisting}

\begin{comment}

\section{Rularea proiectului din STS, pe CloudFoundry (NERECOMANDAT)}

\paragraph{!!! ATENTIE !!!} 
Accesul la acest cont/server este shared (e folosit inclusiv de catre Jenkins), 
pot aparea suprapuneri nedorite intre variante diferite / dezvoltatori diferiti !!!

Se adauga in view-ul 'Servers' un nou server remote: click-dreapta -> 'New'
-> 'Server' -> vendor 'VMware' -> tip 'Cloud Foundry' -> 'Next'.

Se completeaza wizard-ul conform urmatorilor pasi:
\begin{itemize}
  \item 
Email: roda.devel@gmail.com
  \item 
Parola: RodaAdor
  \item 
'Validate Account'
  \item 
'Next'
  \item 
Se muta proiectul 'roda' din lista 'Available' in lista 'Configured'
  \item 
'Next'
  \item 
In fereastra 'Application details', se selecteaza 'Application Type' = 'Spring'
  \item 
'Next'
  \item 
'Deployed URL': roda.cloudfoundry.com
  \item 
'Memory Reservation': 2048 M
  \item 
'Next'
  \item 
Se bifeaza serviciul 'roda-postgres' (baza de date Postgresql) in lista
aparuta.
  \item 
'Finish'
\end{itemize}

Proiectul este disponibil online permanent la adresa:

\url{http://roda.cloudfoundry.com}

\end{comment}

\section{Integrarea continua a aplicatiei}

\subsection{Instalarea si configurarea Jenkins}

Lista plugin-urilor Jenkins ce trebuie instalate:
\begin{itemize}
\item 
Ant Plugin
\item
conditional-buildstep
\item
Credentials Plugin
\item
External Monitor Job Type Plugin
\item
Flexible Publish Plugin
\item
GitHub API Plugin
\item
GitHub plugin
\item
Javadoc Plugin
\item
Jenkins CVS Plug-in
\item
Jenkins Email Extension Plugin
\item
Jenkins GIT client plugin
\item
Jenkins GIT plugin
\item
Jenkins Mailer Plugin
\item
Jenkins SSH Slaves plugin
\item
Jenkins Subversion Plug-in
\item
Jenkins Translation Assistance plugin
\item
Jenkins VirtualBox Plugin
\item
Jenkins Workspace Cleanup Plugin
\item
LDAP Plugin
\item
Maven 2 Project Plugin
\item
pam-auth
\item
Run Condition Plugin
\item
SSH Credentials Plugin
\item
Token Macro Plugin
\end{itemize}

% TODO ???
\begin{comment}
The installer sets up Jenkins as a launch daemon, listening on port 8080. (If you want to know more about launchd and daemons, see here and here.)
Changing boot configuration

The launch daemon picks up the command line options from a standard preferences file, /Library/Preferences/org.jenkins-ci.plist. If the file does not exist, built-in defaults are used. The preference files are manipulated using the standard utility defaults.

    To view all settings in the file, run: defaults read /Library/Preferences/org.jenkins-ci
    To get the value of a single setting, run: defaults read /Library/Preferences/org.jenkins-ci SETTING
    To set the value of a setting, run: defaults write /Library/Preferences/org.jenkins-ci SETTING VALUE
    For more information, see man defaults

Supported Settings

The list of settings supported by the Jenkins launch daemon (see documentation):

    prefix
    httpPort
    httpListenAddress
    httpsPort
    httpsListenAddress
    ajp13Port
    ajp13ListenAddress

Additionally, you can set also these:

    war (Full path name to jenkins.war file.)
    heapSize (Passed to java command-line -Xmx parameter.)
    permGen (Passed to java command-line --XX:MaxPermSize parameter.)
    JENKINS_HOME (Full path to JENKINS_HOME directory where Jenkins keeps its files)

Starting/stopping the service
    To manually start the daemon: sudo launchctl load /Library/LaunchDaemons/org.jenkins-ci.plist
    To manually stop the daemon: sudo launchctl unload /Library/LaunchDaemons/org.jenkins-ci.plist

Inheriting your existing Hudson/Jenkins installation
    If you'd like your new installation to take over your existing Jenkins/Hudson data, copy your old data directory into the new JENKINS_HOME directory.
\end{comment}

\subsection{Instalarea si configurarea dependentelor folosite de job-urile Jenkins}
%TODO ??? ruby gems, vmc

% TODO ??? Chrome, Firefox, Firebug, Postman

