La specificarea in instructiunile de mai jos a sistemului de operare:
\begin{description} 
\item [MacOSX =] OS X version 10.8.4, 64 bit
\item [Ubuntu =] Ubuntu 12.04LTS, kernel 3.2.0-39-generic-pae, 32bit
\item [OpenSUSE = ] OpenSUSE 12.2
\item [Postgresql =] Postgresql 9.2.4
\item [Hibernate =] Hibernate 4.1.8
\item [DbWrench =] DbWrench 2.3.2
\item [STS =] Spring Tool Suite 3.4.0.RELEASE - based on Eclipse Juno 3.8.2
\item [Maven =] Maven 3
\item [JVM =] Java Virtual Machine (JVM) 1.6
\item [Jenkins =] Jenkins 1.522
\item [Solr =] Solr 4.4
\item [Sencha Architect =] Sencha Architect 2.2.2
\end{description}

\section{Instalare si configurare - servicii si software aditional}

\subsection{Instalarea pachetelor necesare}
\begin{description}
\item[MacOSX:]

Se instaleaza MacPorts si toate dependentele necesare 
(instructiuni complete la:
\url{http://www.macports.org/install.php} ).

Se ruleaza urmatoarele comenzi, din shell:
\begin{lstlisting}[breaklines=true]
sudo port install maven3 postgresql92 pgadmin3  apache-solr4 R htmldoc doxygen graphviz firefox-x11
\end{lstlisting}

\item[Ubuntu:]
Se ruleaza urmatoarele comenzi, din shell:
\begin{lstlisting}[breaklines=true]
sudo apt-get install openjdk-6-jdk xvfb maven postgresql-server pgadmin3 solr r-base r-base-dev perl htmldoc doxygen graphviz firefox
\end{lstlisting}

\begin {comment}
\item[OpenSUSE:]
% TODO ???
Maven se instaleaza manual sau dintr-un repository non-standard.

R se instaleaza manual sau dintr-un repository non-standard.

Se ruleaza urmatoarele comenzi, din shell:
\begin{lstlisting}[breaklines=true]
sudo zypper in openjdk-6-jdk postgresql pgadmin3 perl htmldoc doxygen graphviz firefox
\end{lstlisting}
\end{comment}

\end{description}

\subsection{LaTeX}
O distributie LaTeX este necesara pentru generarea documentatiei:

\begin{description}
\item[Ubuntu:] 
\begin{lstlisting}[breaklines=true]
	sudo apt-get install texlive
\end{lstlisting}
\item[MacOSX:] 
MacTeX, instalata de la \url{http://tug.org/mactex/} 
\end{description}

Optional, se pot instala editoare de LaTeX - de ex., 
in MacOSX  se poate utiliza TeXShop 
( \url{http://pages.uoregon.edu/koch/texshop/} ).

\subsection{Maven si Java Virtual Machine (JVM)}
Se creeaza in directorul \emph{\$HOME} al utilizatorului un fisier numit
'.mavenrc' 
(ce contine setarile generale pentru Maven, in acest caz legate de memoria utilizata);
in fisier se scrie:
\begin{lstlisting}[breaklines=true]
	MAVEN_OPTS='-Xmx1024m -XX:MaxPermSize=256m'
	export MAVEN_OPTS
\end{lstlisting}

\label{java_version}
Se verifica instalarea Java (e necesar minim Java 1.6) si tipul de JVM
utilizat (32 sau 64 bit), in shell:
\begin{lstlisting}
	java -version
\end{lstlisting}

\subsection{Perl si Selenium}
\paragraph{In MacOSX:}
Se instaleaza modulul Selenium din Perl, cu urmatoarele comenzi in shell si apoi in shell-ul CPAN
(se alege in timpul instalarii site-mirror-ul folosit;
se iese la final din shell-ul CPAN cu Ctrl+D):
\begin{lstlisting}
	sudo /usr/bin/cpan
		install Test::WWW::Selenium
\end{lstlisting}

\subsection{rJava}
Se instaleaza rJava in R, cu urmatoarele comenzi in shell si apoi in shell-ul R
(se alege in timpul instalarii site-mirror-ul folosit;
se iese la final din shell-ul R cu Ctrl+D):
\begin{lstlisting}
	sudo R CMD javareconf
	R
		install.packages('rJava')
\end{lstlisting}

\subsection{DbWrench}
Se instaleaza DbWrench de pe site-ul producatorului:
\url{http://www.dbwrench.com}.

\section{Instalarea si configurarea IDE-ului si a proiectului}

\subsection{Instalare Spring Tool Suite (STS)}
Descarcati STS de pe pagina:
\url{http://www.springsource.org/downloads/sts-ggts},
facand click direct pe link-ul \emph{'Just take me to the download page'}.

Alegeti pentru descarcarea versiunea STS bazata pe Eclipse 3.8.2:
\begin{description}
\item[MacOSX, 64 bit:] 
\begin{lstlisting}
spring-tool-suite-3.3.0.RELEASE-e3.8.2-macosx-cocoa-x86_64-installer.dmg
\end{lstlisting}
\item[Ubuntu, 32 bit:]
% TODO
\end{description}

Instalati STS cu optiunile default (path-ul default este in \emph{\$HOME/springsource} ):
'Next' -> ... -> 'Finish'.

\subsection{Configurarea STS inainte de rulare}
In fisierul \emph{STS.ini} (in Ubuntu/OpenSUSE, acesta este in folder-ul
radacina al STS, ales la pasul anterior; in MacOS X, este in sub-folder-ul
'STS.app/Contents/MacOS/'), se modifica memoria maxima utilizabila de STS.

In \emph{STS.ini}, linia care incepe cu \emph{-Xmx} 
(memoria maxima utilizabila de STS) 
trebuie modificata astfel:
\begin{itemize} 
\item daca JVM este pe 64 biti:
\begin{lstlisting}
	-Xmx3072m
\end{lstlisting}
\item daca JVM este pe 32 biti:
\begin{lstlisting}
	-Xmx1536m
\end{lstlisting}
\end{itemize}

\subsection{Configurarea plugin-urilor si extensiilor STS}

Porniti STS, eventual din linia de comanda, in folder-ul radacina al instalarii STS
(de exemplu: 'springsource/sts-3.3.0.RELEASE' ):
\begin{lstlisting}	
	./STS &
\end{lstlisting}

Din Dashboard (meniu Help -> Dashboard), click pe tab-ul 'Extensions'
(pozitionat stanga-jos, in Dashboard), bifati in lista si apoi instalati
(butonul 'Install', dreapta-jos) urmatoarele extensii:
\begin{itemize}
\item
Cloud Foundry Eclipse Integration
\item
Edgewall Trac
\end{itemize} 

Restartati STS.

Instalati in STS ultima versiune de Subclipse 
(cu suport pentru Subversion versiunea 1.7),
prin meniu 'Help' -> 'Install New Software':
\begin{itemize}
\item 
nume = Subclipse 1.8
\item
URL = \url{http://subclipse.tigris.org/update_1.8.x}
\end{itemize}

Selectati si instalati de la acest update-site toate seturile de plugin-uri din lista.

Restartati STS.

Configurati Subclipse pentru a utiliza SVNKIT:
\begin{enumerate}
\item 
click pe meniu 'Window' -> 'Preferences' -> alegere din lista 'Team' -> 'SVN'
(pe MacOSX: meniu 'Spring Tool Suite' -> 'Preferences' -> 'Team' -> 'SVN')
\item
la 'SVN Interface' (mai jos in pagina de optiuni aparuta) se alege Clientul
folosit:

'SVNKit (pure Java) SVNKit v1.7....'
\end{enumerate}

(Optional) Instalati plugin-ul IDE pentru Perl, prin meniu 'Help' -> 
'Install New Software':
\begin{itemize}
\item 
nume = Perl EPIC
\item
URL = \url{http://e-p-i-c.sf.net/updates/testing}
\end{itemize}

\subsection{Configurare Task Repository [OPTIONAL]}
In STS, in view-ul 'Task List', butonul 'Add Repository', de tipul 'Trac', 
iar in pasul urmator:
\begin{itemize}
\item 
Server URL:    \url{http://fisiere.dyndns.org:8888/roda}
\item
Debifare optiune 'Anonymous'
\item
Completare User si Parola (specifice)
\item
Bifare 'Save password'
\item
'Validate Settings'
\item
'Finish'
\end{itemize}

La ultimul pas, se adauga un Query legat de acest Task-Repository (de exemplu,
se completeaza doar un nume - precum 'ALL' - si se apasa 'Finish').

\subsection{STS Preferences}
In fereastra 'Preferences' a STS:
\begin{itemize}
\item
in Java -> Editor -> Save Actions: se bifeaza 'Perform the selected actions on save' si ' Format source code' (format all lines); se debifeaza 'Organize imports'.
\item
in Java -> Code Style -> Formatter, se selecteaza profilul 'Eclipse [built-in]', se apasa 'Edit', in tab-ul 'Line Wrapping' , campul 'Maximum line width' se completeaza cu '120' (in loc de '80').
Se completeaza in partea de sus numele profilului: 'RODA' , se apasa 'OK'. Apoi, in fereastra 'Preferences' se selecteaza ca 'Active Profile' profilul creat ('RODA').
\item
(optional) in Java -> Appearance -> Members Sort Order, se bifeaza 'Sort member in same category by visibility'.
\item
(optional) in SpringSource -> Dashboard, se debifeaza 'Show Dashboard on Startup'.
\end{itemize}

\subsection{Preluare proiecte in STS din repository}
In STS, adaugati un SVN repository 
(in perspectiva 'SVN Repository Exploring'), 
folosind URL: \url{svn://fisiere.dyndns.org/roda} .

Faceti \emph{Checkout} succesiv din SVN repository 
(click-dreapta pe numele proiectului / directorului, apoi 'Checkout') 
pentru:
\begin{description}
\item [roda:] aplicatia web Java / Spring
\item [dbwrench2-files:] fisiere DbWrench (format XML + SQL), scripturi pentru operatii asupra bazei de date
\end{description}

\begin{comment}
\item [(optional) RODA-Model:] modelul Perl
\end{comment}

\begin{comment}
In STS, se face upgrade la Subversion 1.7 (pt. working copy) pentru fiecare din proiectele de mai sus: 
in perspectiva 'Spring', 
in view-ul 'Package Explorer' sau in view-ul 'Navigator', 
click-dreapta pe numele proiectului respectiv,
si apoi in meniul aparut:
'Team' -> 'Upgrade' -> 'OK'. 
Poate aparea un mesaj de eroare/warning, care indica
faptul ca working-copy local este deja conform versiunii 1.7 a SVN.
\end{comment}

\section{Configurarea locala a proiectului}

\subsection{Instalarea si configurarea serverului Postgresql local}
\label{postgresql_configurare}

\begin{description}
\item[MacOSX:]

Se instaleaza si ruleaza \emph{Postgres.app} (descarcat de la \url{http://postgresapp.com} ).
Se bifeaza in meniul programului optiunea 'Automatically Start at Login'.

\item[orice alt sistem de operare:]

Se configureaza Postgresql ca server pe statia locala 
(preferabil, pornit automat ca serviciu la bootare). 
Se pot configura in fisierele de configurare ale Postgresql tipurile de autentificare ale clientilor la server 
(de ex. pentru a permite accesul bazat pe parola de pe localhost / reteaua locala; sau pentru a permite si accesul prin
internet, nu doar de la statia locala / reteaua locala).
% TODO mai precis ???
\end{description}

In shell, in directorul 'dbwrench-files' (dupe check-out-ul din SVN), 
se ruleaza urmatoarea comanda
(pentru crearea utilizatorului, a bazei de date si a schemelor auxiliare):
\begin{lstlisting}
	./create-roda-db.sh
\end{lstlisting}

\subsection{Configurarea serverului Solr local}
\label{solr_configurare}

In MacOSX, Solr se porneste din shell cu comanda:
\begin{lstlisting}[breaklines=true]
	sudo solr4
\end{lstlisting}
serverul fiind disponibil la URL: \url{http://localhost:8983/solr/}

Pentru configurarea Solr este folosita schema din subfolder-ul 'examples'. 
% TODO mai precis ???

\subsection{Fisierele de configurare ale aplicatiei}
Toate detaliile necesare se regasesc in sectiunea
dedicata fisierelor de configurare din capitolul dedicat componentelor/arhitecturii aplicatiei: 
\ref{fisiere_configurare}.

Configurarile \emph{default} din aceste fisiere nu trebuie neaparat
schimbate, daca se urmeaza intocmai acest ghid de instalare/configurare.

\subsection{Compilarea pentru prima data a proiectului}

In shell, in folderul-radacina al proiectului 'roda':
\begin{lstlisting}
	mvn -e -X clean package
\end{lstlisting}
Toate pachetele necesare sunt descarcate atunci cand se face prima compilare -- care poate dura mai mult.

In STS, se alege perspectiva 'Spring', 
apoi click-dreapta in view-ul 'Package Explorer' 
pe numele proiectului ('roda'), 
apoi din submeniul 'Maven' se aleg succesiv optiunile:
\begin{enumerate}
\item
'Update Project', apoi click 'OK'.
\item
Download JavaDoc
\item
Download Sources
\end{enumerate}

% STS va deschide un view numit 'Roo shell' unde se pot da comenzi Roo referitoare la proiect.

\subsection{Configurarea Serverului de Aplicatii local}
In STS, se trage (drag-and-drop) proiectul 'roda' din view-ul 'Package Explorer' 
peste serverul local prezent in view-ul 'Servers' (VMware vFabric tc Server ...).

\section{Rularea locala a proiectului}

De fiecare data, inainte de pornirea aplicatiei, trebuie sa fie deja pornite:
\begin{description}
\item[Postgresql] 
ca un server local pe portul 5432, avand deja create: 
user-ul, baza de date, schemele (vezi \ref{postgresql_configurare})
\item[Solr] 
ca un server local pe portul 8983, folosind Jetty (vezi \ref{solr_configurare})
\end{description}

\subsection{Rularea din STS, pe Server Local}
Se (re-)porneste serverul de aplicatii local folosind butoanele din view-ul
'Servers'.

Daca portul local 8080 era liber (portul dorit se poate edita dupa dublu-click pe
server, in lista din view-ul 'Servers'), aplicatia e disponibila la:

\url{http://localhost:8080/roda}

\subsection{Rularea pe un Server Local ad-hoc de tip Tomcat}
Din shell, in folder-ul radacina al proiectului 'roda', cu comanda:
\begin{lstlisting}
	mvn -Dmaven.tomcat.fork=false tomcat:run
\end{lstlisting}
Daca portul local 8080 era liber (portul se poate schimba in 'pom.xml'),
aplicatia e disponibila la:

\url{http://localhost:8080/roda}

Serverul se opreste din shell, de ex. cu Ctrl + C.

\subsection{Rularea locala a Selenium Server}
Din shell, in folder-ul radacina al proiectului 'roda', cu comanda:
\begin{lstlisting}
	mvn selenium:start-server
\end{lstlisting}
Selenium Server porneste pe portul local 4444.

\subsection{Rularea automata a tuturor testelor de integrare}
Din shell, in folder-ul radacina al proiectului 'roda', cu comanda:
\begin{lstlisting}
	mvn verify
\end{lstlisting}
Astfel, sunt pornite/oprite automat:
\begin{itemize}
\item
Selenium Server
\item
Xvfb (pentru ecran virtual, folosit de Selenium)
\item
Tomcat (si aplicatia web)
\end{itemize}

%TODO trebuie facut start/stop al Postgresql + Solr

Sunt rulate atat testele scrise in Java 
(folosind Selenium Server si Selenium WebDriver; browser: Mozilla Firefox), 
cat si cele scrise in Perl (folosind Selenium Server).

\subsection{Rularea testelor de integrare scrise in Perl}
Trebuie pornite inainte serviciile necesare - din shell, cu comanda:
\begin{lstlisting}[breaklines=true]
	mvn tomcat:run selenium:xvfb selenium:start-server
\end{lstlisting}
Apoi, se ruleaza testele in Perl, din shell, in 'src/test/perl' :
\begin{lstlisting}[breaklines=true]
	/usr/bin/perl selenium.pl
\end{lstlisting}


\begin{comment}
\subsection{Rularea automata a testelor web (Selenium)}
Din shell, in folder-ul radacina al proiectului 'roda', cu comanda:
\begin{lstlisting}
	mvn tomcat:run selenium:selenese
\end{lstlisting}
\end{comment}


\begin{comment}
\section{Rularea proiectului din STS, pe CloudFoundry (NERECOMANDAT)}

\paragraph{!!! ATENTIE !!!} 
Accesul la acest cont/server este shared (e folosit inclusiv de catre Jenkins), 
pot aparea suprapuneri nedorite intre variante diferite / dezvoltatori diferiti !!!

Se adauga in view-ul 'Servers' un nou server remote: click-dreapta -> 'New'
-> 'Server' -> vendor 'VMware' -> tip 'Cloud Foundry' -> 'Next'.

Se completeaza wizard-ul conform urmatorilor pasi:
\begin{itemize}
  \item 
Email: roda.devel@gmail.com
  \item 
Parola: RodaAdor
  \item 
'Validate Account'
  \item 
'Next'
  \item 
Se muta proiectul 'roda' din lista 'Available' in lista 'Configured'
  \item 
'Next'
  \item 
In fereastra 'Application details', se selecteaza 'Application Type' = 'Spring'
  \item 
'Next'
  \item 
'Deployed URL': roda.cloudfoundry.com
  \item 
'Memory Reservation': 2048 M
  \item 
'Next'
  \item 
Se bifeaza serviciul 'roda-postgres' (baza de date Postgresql) in lista
aparuta.
  \item 
'Finish'
\end{itemize}

Proiectul este disponibil online permanent la adresa:

\url{http://roda.cloudfoundry.com}

\end{comment}

\section{Generarea documentatiei proiectului}

Pentru generarea in directorul 'target/' a documentatiei proiectului in format PDF (din sursele LaTeX):

\begin{lstlisting}
	mvn latex:latex
\end{lstlisting}

Pentru a forta re-generarea tuturor documentelor in format PDF, se sterge directorul 'target/latex/' din proiect:

\begin{lstlisting}
	rm -rf target/latex
\end{lstlisting}

Pentru generarea in directorul 'target/site/apidocs' a documentatiei JavaDoc a proiectului:

\begin{lstlisting}
	mvn javadoc:javadoc
\end{lstlisting}

Pentru generarea in directorul 'target/site' a website-ului proiectului in format HTML (cuprinzand JavaDoc, informatii si rapoarte):

\begin{lstlisting}
	mvn site
\end{lstlisting}

\section{Web Development}
Se instaleaza Mozilla Firefox; in Firefox se instaleaza urmatoarele add-on-uri:
\begin{description}
\item [Firebug]
\item [Selenium IDE]
\end{description}

Se instaleaza Google Chrome; in Chrome se instaleaza:
\begin{description}
\item [Postman] de la \url{http://www.getpostman.com/}
\end{description}


\section{Integrarea continua a aplicatiei}

\subsection{Instalarea si configurarea Jenkins}

Se instaleaza Jenkins de la \url{http://jenkins-ci.org} : 
din sectiunea 'Long-Term Support Release', 
se salveaza si se instaleaza pachetul nativ pentru sistemul de operare.

Pentru a seta portul folosit de Jenkins la altul decat 8080 (de exemplu 9999), se dau in shell comenzile:
\begin{lstlisting}[breaklines=true]
sudo defaults write /Library/Preferences/org.jenkins-ci httpPort 9999
\end{lstlisting}

Pentru a porni / opri manual serviciul Jenkins se folosesc in shell comenzile: 
\begin{lstlisting}[breaklines=true]
sudo launchctl load /Library/LaunchDaemons/org.jenkins-ci.plist
sudo launchctl unload /Library/LaunchDaemons/org.jenkins-ci.plist
\end{lstlisting}

User-ul folosit de Jenkins pentru rularea job-urilor este de obicei numit 'jenkins'.

Daca se doreste migrarea datelor/setarilor/plugin-urilor/job-urilor existente dintr-o alta instalare Jenkins, trebuie copiat directorul indicat de variabila de mediu JENKINS\_HOME de pe vechiul server -> in directorul indicat de variabila de mediu JENKINS\_HOME (de pe noul server); in cazul MacOSX, JENKINS\_HOME = /Users/Shared/Jenkins/Home; directorul trebuie sa fie accesibil pentru utilizatorul folosit de Jenkins pentru rularea job-urilor (de obicei numit 'jenkins'). In configurarile Jenkins, dupa pornirea acestuia,  trebuie actualizate: URL-ul la care este accesibil Jenkins.

\subsection{Instalarea si configurarea plugin-urilor Jenkins}
Lista plugin-urilor Jenkins ce trebuie instalate:
\begin{itemize}
\item 
Ant Plugin
\item
conditional-buildstep
\item
Credentials Plugin
\item
External Monitor Job Type Plugin
\item
Flexible Publish Plugin
\item
GitHub API Plugin
\item
GitHub plugin
\item
Javadoc Plugin
\item
Jenkins CVS Plug-in
\item
Jenkins Email Extension Plugin
\item
Jenkins GIT client plugin
\item
Jenkins GIT plugin
\item
Jenkins Mailer Plugin
\item
Jenkins SSH Slaves plugin
\item
Jenkins Subversion Plug-in
\item
Jenkins Translation Assistance plugin
\item
Jenkins VirtualBox Plugin
\item
Jenkins Workspace Cleanup Plugin
\item
LDAP Plugin
\item
Maven 2 Project Plugin
\item
pam-auth
\item
Run Condition Plugin
\item
SSH Credentials Plugin
\item
Token Macro Plugin
\end{itemize}

\begin{comment}
The installer sets up Jenkins as a launch daemon, listening on port 8080. (If you want to know more about launchd and daemons, see here and here.)
Changing boot configuration

The launch daemon picks up the command line options from a standard preferences file, /Library/Preferences/org.jenkins-ci.plist. If the file does not exist, built-in defaults are used. The preference files are manipulated using the standard utility defaults.

    To view all settings in the file, run: defaults read /Library/Preferences/org.jenkins-ci
    To get the value of a single setting, run: defaults read /Library/Preferences/org.jenkins-ci SETTING
    To set the value of a setting, run: defaults write /Library/Preferences/org.jenkins-ci SETTING VALUE
    For more information, see man defaults

Supported Settings

The list of settings supported by the Jenkins launch daemon (see documentation):

    prefix
    httpPort
    httpListenAddress
    httpsPort
    httpsListenAddress
    ajp13Port
    ajp13ListenAddress

Additionally, you can set also these:

    war (Full path name to jenkins.war file.)
    heapSize (Passed to java command-line -Xmx parameter.)
    permGen (Passed to java command-line --XX:MaxPermSize parameter.)
    JENKINS_HOME (Full path to JENKINS_HOME directory where Jenkins keeps its files)

Starting/stopping the service
    To manually start the daemon: 
    sudo launchctl load /Library/LaunchDaemons/org.jenkins-ci.plist
    
    To manually stop the daemon: 
    sudo launchctl unload /Library/LaunchDaemons/org.jenkins-ci.plist

Inheriting your existing Hudson/Jenkins installation
    If you'd like your new installation to take over your existing Jenkins/Hudson data, copy your old data directory into the new JENKINS_HOME directory.
\end{comment}

\begin{comment}

x11vnc, de la http://www.karlrunge.com/x11vnc/ :  x11vnc-0.9.13.tar.gz
pasii de la instructiunile de instalare +
    ./configure --without-x --with-jpeg=/opt/local
\end{comment}


\subsection{Instalarea unor pachete necesare job-urilor Jenkins}
\begin{description}
\item[MacOSX:]
Se ruleaza urmatoarele comenzi, din shell:
\begin{lstlisting}[breaklines=true]
sudo port install rb-rubygems
sudo gem install cf
\end{lstlisting}

\item[Ubuntu:]
Se ruleaza urmatoarele comenzi, din shell:
\begin{lstlisting}[breaklines=true]
sudo apt-get install rubygems
sudo gem install cf
\end{lstlisting}
\end{description}
