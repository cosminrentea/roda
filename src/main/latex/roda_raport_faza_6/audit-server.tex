\subsection{Servicii de jurnalizare pe partea de server}

In etapa precedenta a proiectului a fost realizat sistemul de jurnalizare al aplicatiei. Detaliile referitoare la tehnologia folosita si la implementarea acesteia au fost descrise in anexa raportului corespunzator etapei a patra a proiectului.

Ulterior implementarii sistemului de jurnalizare, au fost dezvoltate controller-e si servicii care permit obtinerea de informatii relevante in urma procesului de audit. Un scop important al noilor clase dezvoltate il constituie furnizarea acestor date interfetei de administrare a aplicatiei, pentru o buna monitorizare a activitatii care se va desfasura pe platforma RODA.

Controller-ele dezvoltate permit lansarea de cereri sub forma URL-uri de mai jos:

\begin{itemize}
\item{/admin/revised-objects - furnizeaza lista obiectelor care au aparut in cadrul reviziilor asociate procesului de audit al aplicatiei}
\item{/admin/revised-users - furnizeaza lista utilizatorilor (campul username) care au efectuat revizii}
\item{/admin/revised-dates - furnizeaza lista datelor calendaristice la care s-au efectuat revizii}
\item{/admin/revisions[/{revision}] - furnizeaza informatii asupra tuturor reviziilor daca partea optionala a URL-ului nu este prezenta, respectiv asupra reviziei avand identificatorul egal cu valoarea lui fragmentului revision din URL, in caz contrar.}
\item{/admin/revisions-by-date[/{date}] - furnizeaza informatii despre revizii, grupate dupa datele la care acestea au avut loc. Daca fragmentul date al URL-ului este prezent, se obtin informatii despre reviziile desfasurate la data respectiva. In particular, in interfata este utila folosirea controller-ului corespunzator lui /admin/revised-dates, pentru obtinerea tuturor datelor posibile, urmata de selectarea unei date din lista returnata. Pentru data respectiva, are loc invocarea acestui controller si obtinerea informatiilor despre reviziile corespunzatoare ei.}
\item{/admin/revisions-by-object[/{object}] - furnizeaza informatii despre revizii, grupate dupa obiectele asupra carora acestea au avut loc. Daca fragmentul object al URL-ului este prezent, se obtin informatii despre reviziile in care au avut loc modificari asupra obiectului respectiv. In particular, in interfata este utila folosirea controller-ului corespunzator lui /admin/revised-objects, pentru obtinerea tuturor obiectelor posibile, urmata de selectarea unui obiect din lista returnata. Pentru obiectul respectiv, are loc invocarea acestui controller si obtinerea informatiilor despre reviziile care au l-au afectat.}
\item{/admin/revisions-by-user[/{username}] - furnizeaza informatii despre revizii, grupate dupa utilizatorii care le-au efectuat. Daca fragmentul username al URL-ului este prezent, se obtin informatii despre reviziile efectuate de catre utilizatorul respectiv. In particular, in interfata este utila folosirea controller-ului corespunzator lui /admin/revised-users, pentru obtinerea tuturor utilizatorilor posibili, urmata de selectarea unui utilizator din lista returnata. Pentru utilizatorul respectiv, are loc invocarea acestui controller si obtinerea informatiilor despre reviziile pe care le-a efectuat.}
\item{"/admin/revisionsinfo[/{revision}] - furnizeaza informatii complete asupra tuturor reviziilor daca partea optionala a URL-ului nu este prezenta, respectiv asupra reviziei avand identificatorul egal cu valoarea lui fragmentului revision din URL, in caz contrar. In particular, in interfata este utila folosirea controller-ului corespunzator lui /admin/revisions, pentru obtinerea tuturor reviziilor posibile, urmata de selectarea unei revizii din lista returnata. Pentru revizia respectiva, are loc invocarea acestui controller si obtinerea informatiilor referitoare la obiectele, liniile si coloanele modificate, data reviziei, precum si utilizatorul asociat ei.}
\item{/admin/simple-revisions-by-object[/{object}[/{id}]] - furnizeaza informatii generale (simplificate) asupra reviziilor desfasurate asupra tuturor obiectelor (daca nu este specificata partea optionala a URL-ului), asupra unui singur obiect (daca este specificat fragmentul object din URL) sau asupra unei linii a unui obiect (daca sunt specificate atat fragmentele object si id).}
\item{/admin/simple-revisions-by-user[/{username}] - furnizeaza informatii generale (simplificate) asupra reviziilor desfasurate de catre fiecare utilizator daca partea optionala a URL-ului nu este prezenta, respectiv asupra reviziilor efectuate de catre utilizatorul corespunzator valorii din fragmentului username din URL, in caz contrar.}
\end{itemize}




