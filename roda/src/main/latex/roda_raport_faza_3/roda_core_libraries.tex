\section{Biblioteci nucleu}

Aplicatia RODA respecta principiile unei aplicatii web moderne, de tip MVC. Suplimentar, pe langa cele 3 niveluri specifice unei astfel de aplicatii, RODA are definit un nivel de servicii, care permite izolarea modelului fata de controllere si specificarea de drepturi de acces personalizate (prin liste de control al accesului - ACL). Exista servicii disponibile expuse aplicatiei ca JavaBeans, precum cele pentru: 
indexare si cautare, file-repository, integrarea unui motor statistic, executia asincrona sau programata de task-uri etc. Au fost definite controllere corespunzand unui sablon de interfata de tip \emph{CRUD}
(Create / Read / Update / Delete), pentru fiecare clasa din model. In controllere se fac apeluri catre nivelul de servicii. Au mai fost definite si controllere pentru a servi continut de baza in format json.

Componentele dezvoltate in cadrul etapelor anterioare ale proiectului au fost completate cu functionalitati suplimentare, printre scopurile carora amintim furnizarea anumitor servicii si asigurarea suportului pentru activitatile viitoare. In acest sens, in clasele domeniului au fost implementate metode dintre care mentionam: verificarea egalitatii dintre instante, generarea unor valori hash corespunzatoare datelor stocate in cadrul instantelor si testarea existentei unor entitati in baza de date. Pe langa utilitatea lor in cadrul etapei curente, aceste metode vor servi ulterior componentei care va realiza importul de studii in baza de date si vor asigura unicitatea datelor in raport cu modelul conceptual. 

Dintre elementele adaugate in cadrul nucleului aplicatiei se disting controller-erele si serviciile care asigura legatura dintre interfata aplicatiei si baza de date. Astfel, au fost implementate metode de recuperare a datelor corespunzatoare claselor domeniului, conform unor cerinte specifice, sub forma unor fisiere de tip json care servesc vizualizarii prin intermediul interfetei, intr-un mod care usureaza intelegerea lor de catre utilizator. Din categoria acestor servicii au fost implementate cele care sunt necesare functionarii componentei DataBrowser a interfetei. Ulterior, aceste servicii vor fi extinse cu cele prevazute in cadrul celorlalte componente ale aplicatiei. Pe langa partea de controller-e ce asigura legatura dintre nivelurile arhitecturale ale aplicatiei, bibliotecile nucleu vor fi extinse cu clase care vor gestiona relatia cu motorul statistic R si cu platforma de cautare Solr.




