In aceasta sectiune descriem aspectele legate de dezvoltarea in practica, intr-o
maniera incrementala si cu asigurarea calitatii procesului de dezvoltare.

\section{Documentare}
Se folosesc in codul-sursa comentarii de tip Javadoc. 

In procesul de build se genereaza automat documentatie a codului de tip HTML; 
aceasta include si artefacte generate de UMLGraph (diagrame de clase).
Pornind de la aceasta documentatie in format HTML, se poate genera si documentatie in format PDF 
folosind utilitarul \emph{htmldoc}.

Documentatia referitoare la proiect se genereaza sub forma unui site tipic Maven, cu comanda
\begin{lstlisting}
	mvn site
\end{lstlisting}
Acest site generat contine informatii si rapoarte despre proiect, astfel:
\begin{description}

\item [Project Information] :

\begin{itemize}
\item    About
\item    Plugin Management
\item    Distribution Management
\item    Dependency Information
\item    Source Repository
\item    Mailing Lists
\item    Issue Tracking
\item    Continuous Integration
\item    Project Plugins
\item    Project License
\item    Project Team
\item    Project Summary
\item    Dependencies
\end{itemize}

\item [Project Reports] :

\begin{itemize}
\item    JavaDocs
\item    Test JavaDocs
\item    Source Xref
\item    Test Source Xref
\item    Surefire Report
\item    Failsafe Report
\item    CPD Report
\item    PMD Report
\item    FindBugs Report
\item    Tag List
\item    JDepend
\item    JavaNCSS Report
\item    Third Parties
\end{itemize}
\end{description}

Rapoartele de proiect includ printre altele: 
comentariile Javadoc, 
codul sursa si de testare in format cross-reference (hyperlinked), 
rapoarte de teste (de unitati si de integrare),
rapoarte ale potentialelor bug-uri si vulnerabilitati,
metrici legate de design-ul claselor OOP, 
lista dependentelor externe ale proiectului. 

\section{Testarea aplicatiei}

Testarea aplicatiei web la nivel de unitati este realizata folosind JUnit 4
si facilitatile oferite de Spring pentru rularea acestor teste (SpringJUnit4ClassRunner) 
intr-un context care sa includa si configuratia pentru framework.

Testarea aplicatiei web la nivel de integrare este realizata in acest moment prin 4 metode:
\begin{enumerate}
\item
teste JUnit de integrare, ce utilizeaza interfata WebDriver a Selenium pentru a comanda direct browser-ul (Mozilla Firefox)
\item
teste JUnit de integrare, ce interactioneaza cu \emph{Selenium Server}
\item
scripturi de testare scrise in Perl, ce interactioneaza cu \emph{Selenium Server}
\item
o suita de teste pentru interfata web de tip CRUD, 
teste ce au fost generate semi-automat in limbajul \emph{Selenese} 
si executate prin \emph{Selenium}.
\end{enumerate}

\section{Procese de Build}
Procesele de build sunt realizate prin Maven;
fisierul de build aferent (\emph{pom.xml})
ce contine dependentele necesare pentru configurarea, compilarea,
deployment-ul, rularea, testarea, documentarea codului-sursa (Javadoc) si a proiectului(LaTeX).

Serviciul de tip continuous-build (\emph{Jenkins}) 
realizeaza efectiv pasii descrisi mai sus 
atunci cand repository-ul SVN ce contine codul-sursa este modificat. 
%TODO detaliere job-uri Jenkins

In job-ul Jenkins aferent proiectului exista si pasi optionali pentru:
\begin{itemize}
\item
deployment al aplicatiei si remote pe un
serviciu de tip \emph{cloud} (\url{http://cloudfoundry.com}) 
\item
re-crearea schemei curente a bazei de date pe un server Postgresql intern,
pentru referinta.
\end{itemize}

%TODO pot fi generate URL-uri corecte, de ex. catre fisierele de configurare ce sunt hostate pe github ?!

\section{Servicii TIC destinate echipei proiectului}
\begin{description}

\item[File-Server] (AjaXplorer) - pentru rapoarte, manuale, standarde, documente read-only, date legacy etc.

\item[Serviciu de versionare a codului-sursa] (Subversion 1.7) , inclusiv
    cu notificari prin e-mail

\item[Bugtracking si project-management] (Trac), inclusiv notificari prin
    e-mail si integrare cu IDE (Spring Tool Suite)

\item[Backup] - pentru datele aferente celor 3 servicii enumerate mai sus

\item[Lista de e-mailuri] - pentru coordonarea echipei de dezvoltare

\item[Serviciu de integrare continua] (Jenkins) - pentru asamblarea si
    testarea aplicatiei dezvoltate, in mod continuu, cu notificari

\item[Serviciu de indexare/cautare] (Apache Solr) - utilizabil in
    conjunctie cu aplicatia dezvoltata

\item[Document-Management Server] (Alfresco Community Edition) -
    utilizabil in conjunctie cu aplicatia dezvoltata

\item[Database server] (Postgresql) - utilizat in conjunctie cu
    aplicatia dezvoltata

\item[Development \& production web application servers] (Apache Tomcat 7) -
    gazduiesc aplicatia dezvoltata

\item[Server web] (Apache) - in curs de configurare

\item[Servicii de tip Director] (LDAP 3) - in curs de configurare

\item[Servicii de autentificare federata] (Shibboleth / SAML2) - in curs de
configurare

\end{description}
