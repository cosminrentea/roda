Datele initiale ale aplicatiei sunt importate chiar la pornirea serverului folosind un 'hook' al sistemului Spring MVC (ApplicationListener <ContextRefreshedEvent>).

Functionalitatile de importare a datelor sunt grupate intr-un singur serviciu (reprezentat de o interfata si implementare) aflat in pachetul 'ro.roda.importer'.

Serviciul de import permite importarea datelor initiale ale aplicatiei, precum si a celor folosite la testare.

\section{Importarea datelor initiale ale aplicatiei}

Aceste date sunt cele ce vor fi folosite si in versiunea finala de productie a aplicatiei, precum: 
utilizatori si roluri, drepturi de acces standard, limbi, date geografice, thesaurus din domeniul stiintelor sociale etc.

Se importa din mai multe formate:
\begin{itemize}
\item din format CSV (Comma-Separated Values). Se folosesc pentru acest scop fie pachetul OpenCSV, fie API-ul oferit de driverul JDBC al Postgresql pentru a importa fisiere CSV).
\item din format SQL. Se foloseste fisierul 'import.sql', pe care componenta ORM (Hibernate) este configurata sa il importe la initializarea sa.
\item din thesaurus-ul ELSST. Se importa termenii si ulterior relatiile dintre ei din mai multe fisiere in format CSV, 
populandu-se mai multe tabele (cele din zona 'Topics' din schema bazei de date).
\item din formatul SPSS -- pentru datele propriu-zise aferente studiilor descrise prin fisierele DDI; a fost imbunatatit si adaptat un parser inceput de catre Open Data Foundation.
\item din formatul DDI-Codebook (mai multe detalii mai jos)
\end{itemize}

\section{Importarea datelor suplimentare / de testare pentru aplicatie}

Cu datele obtinute din fisiere format CSV sunt populatemai multe tabele mai putin importante, care nu trebuie populate si in versiunea de productie, ci doar in cea de development si testare;
de exemplu, prefixe, sufixe, date despre persoane si organizatii, date folosite de catre sub-sistemul CMS, configurari pentru sub-sistemul de autorizare prin ACL etc.

Un alt rol al acestor date suplimentare este acela de a efectua legaturi, agregari si corecturi de date, ce sunt necesar in special dupa faza de importare a fisierelor DDI legacy.

\section{Detalii despre importarea datelor in format DDI}

Importer-ul proceseaza si salveaza in baza de date fisiere XML ce corespund standardului DDI-Codebook (Data Documentation Initiative, \url{http://www.ddialliance.org/Specification/DDI-Codebook/}).

Se foloseste un pachet de clase Java derivat de catre JAXB pornind de la definitia in format 'XML Schema' a DDI-Codebook; acest pachet se numeste 'ro.roda.ddi'.
Este posibil sa se salveze si din formatul DDI nativ direct in baza de date (practic se parseaza si se construiesc la runtime tabelele corespunzand tag-urilor din formatul DDI).

Cele mai importante tag-uri din fisierele DDI ce sunt procesate sunt:
\begin{description}
\item [docDscr] --
consta in informatii bibliografice descriind documentul conform cu DDI ca un intreg. 
Aceasta sectiune poate fi vazuta ca un "wrapper" / ambalaj al carui elemente descriu in mod unic intreg continutul fisierului conform cu DDI. 
Aceasta sectiune trebuie sa fie cat mai completa posibil, si cuprinde informatii despre 
autor, persoane si organizatii responsabile de continut si drepturile de proprietate intelectuala, producatorul, sursa fisierului (codebook-ul original). 
Astfel, aceste campuri permit creatorului fisierului DDI sa produca descrieri ale versiunii, responsabilitatilor, 
si alte descrieri care au legatura atat cu crearea acelui fisier DDI ca o versiune separata si reformatata a materialelor-sursa (fie ele print sau electronice) 
cat si a materialelor-sursa propriu-zise.

\item [stdyDscr] -- 
informatii despre colectia de date, studiu, sau compilatie pe care o descrie documentatia DDI; 
informatii despre cum ar trebui citat studiul, cine a colectat, corectat, agregat datele, cine distribuie datele, cuvinte-cheie despre continutul datelor, un rezumat al continutului datelor, metode de colectare si procesare a datelor etc.

\item [fileDscr] --
informatii despre fisierele cu date care formeaza o colectie; aceasta sectiune poate fi repetata pentru colectii cu fisiere multiple. 
Aici se pot face si referinte la "summary data description" si la aspecte de metodologie anterior descrise, care se aplica unor anumite fisiere din colectie.

\item [dataDscr] --
descrierea variabilelor (si optional a unor calcule statistice pe baza valorilor acestora).

\item [otherMat] -- 
permite includerea si etichetarea altor materiale care sunt in legatura cu studiul;
sectiunea poate servi si ca un "container" pentru alte materiale in format electronic (de ex. chestionare, note de codificare, SPSS/SAS/Stata fisiere de setup si altele, manuale pentru utilizatori, ghiduri pentru continuarea cercetarii, glosare, instructiuni pentru operatori, harti, scheme de baze de date, dictionare de date, informatii despre valorile-lipsa etc.).

\end{description}

Cele mai importante tabele din baza de date ce sunt populate in cadrul procesului de import sunt cele din
zonele de Studii, Organizatii, Persoane, Adrese, Instante,  Fisiere, Variabile si Chestionare. 

Aceasta componenta software permite preluarea si utilizarea integrala a datelor de tip legacy din arhiva electronica a RODA (50 studii sociale).

Importer-ul din formatul DDI a fost finalizat in aceasta faza a proiectului.

\section{Configurarea importerului}

Importer-ul este configurat prin intermediul fisierului de proprietati 'roda.properties', prin setarea unor chei (avand de exemplu valorile de mai jos) 
pentru a declansa importarea datelor in format CSV, DDI, SPSS:

\begin{lstlisting}
# Data, import etc.
roda.data.csv=yes
#roda.data.csv=no
roda.data.csv.dir=csv/

roda.data.csv-extra=yes
#roda.data.csv-extra=no
roda.data.csv-extra.dir=csv-extra/

#roda.data.elsst=yes
roda.data.elsst=no
roda.data.elsst.dir=elsst/

roda.data.ddi=no
#roda.data.ddi=yes
#roda.data.ddi.persist=yes
roda.data.ddi.persist=no
roda.data.ddi.xsd=xsd/ddi122.xsd

# The file(s) to be imported (regular expressions can be used)

#roda.data.ddi.files=ddi/073_F1.xml
#roda.data.ddi.files=ddi/080_F1.xml
roda.data.ddi.files=ddi/07*_F1.xml
#roda.data.ddi.files=ddi/08*_F1.xml
#roda.data.ddi.files=ddi/*.xml

roda.data.spss=yes

#roda.data.csv-after-ddi=no
roda.data.csv-after-ddi=yes
roda.data.csv-after-ddi.catalog_study=csv-after-ddi/catalog_study.csv
roda.data.csv-after-ddi.series_study=csv-after-ddi/series_study.csv

\end{lstlisting}