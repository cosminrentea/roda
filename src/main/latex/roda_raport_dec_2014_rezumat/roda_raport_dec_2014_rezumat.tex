\documentclass[a4paper, 10pt]{article}
\usepackage[ascii]{inputenc}
\usepackage[T1]{fontenc}
\usepackage[romanian,english]{babel}
\usepackage{amsmath}
\usepackage{amssymb,amsfonts,textcomp}
\usepackage{comment}
\usepackage{color}
\usepackage{array}
\usepackage{hhline}
\usepackage{hyperref}
\hypersetup{pdftex, colorlinks=true, linkcolor=blue, citecolor=blue, filecolor=blue, urlcolor=blue, pdftitle=, pdfauthor=, pdfsubject=, pdfkeywords=}
\usepackage[pdftex]{graphicx}

%%%% Cosmin

\usepackage[a4paper, margin=2.3cm]{geometry}
\renewcommand*{\familydefault}{\sfdefault} %Sans Serif font
\renewcommand{\sfdefault}{phv} % Arial
\setlength{\parindent}{0pt} % no indentation for paragraphs
\setlength{\tabcolsep}{70pt} % table inter-column spacing

%%%%

% List styles
\newcommand\liststyleLv{%
\renewcommand\theenumi{\arabic{enumi}}
\renewcommand\theenumii{\arabic{enumii}}
\renewcommand\theenumiii{\arabic{enumiii}}
\renewcommand\theenumiv{\arabic{enumiv}}
\renewcommand\labelenumi{\theenumi.}
\renewcommand\labelenumii{\theenumii.}
\renewcommand\labelenumiii{\theenumiii.}
\renewcommand\labelenumiv{\theenumiv.}
}
\newcommand\liststyleLSi{%
\renewcommand\theenumi{\arabic{enumi}}
\renewcommand\theenumii{\arabic{enumii}}
\renewcommand\theenumiii{\arabic{enumiii}}
\renewcommand\theenumiv{\arabic{enumiv}}
\renewcommand\labelenumi{\theenumi.}
\renewcommand\labelenumii{\theenumii.}
\renewcommand\labelenumiii{\theenumiii.}
\renewcommand\labelenumiv{\theenumiv.}
}

% Footnote rule
\setlength{\skip\footins}{0.047in}
\renewcommand\footnoterule{\vspace*{-0.0071in}\setlength\leftskip{0pt}\setlength\rightskip{0pt plus 1fil}\noindent\textcolor{black}{\rule{0.25\columnwidth}{0.0071in}}\vspace*{0.0398in}}

% Pages styles
\makeatletter
\newcommand\ps@Standard{
  \renewcommand\@oddhead{}
  \renewcommand\@evenhead{}
  \renewcommand\@oddfoot{}
  \renewcommand\@evenfoot{}
  \renewcommand\thepage{\arabic{page}}
}

\makeatother
\pagestyle{plain}
\title{}
\author{}
\date{2013-04-08}

\begin{document}
{\raggedleft\bfseries
MACHETA 3
\par}

{\bfseries
Contractor: Universitatea din Bucure\c{s}ti}

{\textbf{Cod fiscal: 45055002}

\bigskip


\begin{tabular}{@{}l l}
\textbf{Avizat,}&\textbf{De acord,}\\
\textbf{Comisia de monitorizare}&\textbf{DIRECTOR PLAN SECTORIAL}\\
\\
\textbf{PRE\c{S}EDINTE:}&\\
Rolanda Predescu&\\
\\
\\
\textbf{MEMBRII:}&\textbf{MONITOR PROIECT}\\
Marioara Iordan&Daniela Dinic\u{a}\\
\\
\\
Valentina Vasile&\\
\\
\\
Speran\c{t}a P\^{a}rciog\\
\\
\\
\end{tabular}

\bigskip

\bigskip

{\centering\bfseries
RAPORT DE ACTIVITATE AL FAZEI
\par}

\bigskip

{\bfseries
Contractul nr.: 5S / 27.07.2012}

{
\textbf{Proiectul: }
\textit{`` Sistem informatic integrat pentru identificarea, arhivarea \c{s}i diseminarea bazelor de date \c{s}i a indicatorilor din
cercet\u{a}rile sociale ''}}

%TODO Numarul fazei !
%TODO Titlul fazei ! ???
{
\textbf{Faza: }
Nr. 7 cu titlul 
\textit{`` ????????? ''}}

{\textbf{Termen:} 10.12.2014}

\medskip

\section{Obiectivul proiectului}

Realizarea unei arhive electronice integrate care s\u{a}
con\c{t}in\u{a} \c{s}i s\u{a} distribuie c\^at mai multe dintre 
datele sociologice acumulate \^in Rom\^ania.

\medskip

Sistemul trebuie s\u{a} ofere cercet\u{a}torilor \^in domeniul \c{s}tiin\c{t}elor sociale instrumentele necesare pentru
trecerea \^in revist\u{a}, compararea, sintetizarea, ad\u{a}ugarea datelor sociologice de interes. 
Operatorii interni ai arhivei vor c\u{a}uta, 
identifica \c{s}i acumula date sociologice disponibile \^in Rom\^ania.

\medskip

Arhiva va fi integrat\u{a} \^in Consiliul European al Arhivelor de Date Sociale (CESSDA) asigur\^andu-se un schimb
continuu bidirec\c{t}ional de informa\c{t}ie.

\section{Rezultate preconizate pentru atingerea obiectivului}

\begin{enumerate}
\item {
Sistem informatic de arhivare (stocare, catalogare plus proceduri \c{s}i capacit\u{a}\c{t}i de identificare \c{s}i
accesare) \foreignlanguage{romanian}{\c{s}}i diseminare a datelor sociale produse de pia\c{t}a cercet\u{a}rii sociale
din Rom\^ania.}
\item {
Asigurarea procedurilor de securizare a accesului la datele arhivate, ca urmare a investi\c{t}iilor \^in hardware
\foreignlanguage{romanian}{\c{s}}i software pe parcursul proiectului;}
\item {
\foreignlanguage{romanian}{Arhiva }va func\c{t}iona inclusiv ca o banc\u{a} de date sociale, dat fiind faptul c\u{a}
produc\u{a}torii de date nu au nici capacit\u{a}\c{t}ile tehnice nici \foreignlanguage{romanian}{cunoa\c{s}terea}
necesar\u{a} depozit\u{a}rii, catalog\u{a}rii \c{s}i acces\u{a}rii cercet\u{a}rilor realizate, pe termen lung;}
\item {
Facilitarea accesului comunit\u{a}\c{t}ii de cercetare din Rom\^ania la datele produse \^in ultimii 20 de ani pe
pia\c{t}a na\c{t}ional\u{a} de profil, dar \c{s}i la cele europene, ca urmare a conect\u{a}rii arhivei la CESSDA-ERIC
(arhiva fiind deja membru al CESSDA);}
\item {
Facilitarea accesului comunit\u{a}\c{t}ii de cercetare interna\c{t}ionale la datele produse \^in Rom\^ania prin
intermediul CESSDA va aduce de asemenea mari beneficii interna\c{t}ionaliz\u{a}rii cercet\u{a}rii sociale din
Rom\^ania;}
\item {
Articole de specialitate/comunic\u{a}ri \c{s}tiin\c{t}ifice menite a face cunoscute pe plan na\c{t}ional \c{s}i
interna\c{t}ional beneficiile sistemului implementat ca urmare a derul\u{a}rii sale.}
\end{enumerate}

\section{Obiectivul fazei}

%TODO Obiectivul fazei, cf. documentatiei proiectului 


%\begin{itemize}
%\item

%\item

%\end{itemize}

\section{Rezultate preconizate pentru atingerea obiectivului fazei}

%TODO Rezultate ale fazei, cf. documentatiei proiectului

\begin{comment}
 
Urmatoarele sisteme ale aplicatiei au fost realizate:
\begin{itemize}
\item
Modul de administrare
\item
Modul de acces la date
\item
Motor statistic
\end{itemize}

\end{comment}


\section{Rezumatul fazei}

\medskip

\subsection{Servicii web: Cautare}

Pentru a oferi posibilitatea de a cauta intre datele si metadatele disponibile la RODA 
a fost implementata componenta de Cautare.

Aceasta se bazeaza pe un server de indexare/cautare (Apache Solr) ce ruleaza separat de aplicatia web a RODA. 
La importarea initiala a datelor, precum si automat la modificarea anumitor tabele din baza de date 
(de ex. pagini CMS, fisiere CMS, serii / cataloage, studii, intrebari si variabile), 
se trimit date catre serverul Solr care realizeaza indexarea lor.
Cautarea se realizeaza din interfata publica sau cea de administrare si este senzitiva la limba curenta in care este afisata aplicatia.
??? detalii search din client

Schema standard a Solr a fost modificata astfel:
??? 

Astfel, se observa ca toate "documentele" (in terminologia Solr) care sunt indexate au cel putin urmatoarele campuri:
???

Campul "url" permite listarea rezultatelor cautarilor avand link-uri directe asociate.

\subsection{Servicii web: API REST}

Pentru a oferi posibilitatea de a accesa datele aplicatiei si de catre partenerii RODA 
(nu doar din interfata de administrare si cea publica), a fost luata decizia de a expune in mod securizat
un API de tip REST, care sa comunice cu clientii pe web prin mesaje structurate JSON.
Astfel, un partener isi poate defini intr-o modalitate proprie un client (de ex. prin JavaScript, serviciu web)
care sa solicite date granulare din baza de date a RODA.

Principiul pe care l-am aplicat in acest caz a fost HATEOAS. 
??? detalii

A fost folosit un pattern de programare care se bazeaza pe interfete de tip Repository / CrudRepository 
din proiectul Spring Data (sub-proiectele Spring Data JPA si Spring Data REST MVC). 
A fost configurat un URL Dispatcher special care este responsabil de accesarea datelor prin JSON in paradigma HATEOAS. 
Pentru fiecare clasa din domeniu (corespunzatoare cate unui tabel in baza de date) 
a fost definita intr-un pachet separat o clasa ce extinde CrudRepository, 
permitand astfel operatii uzuale - inclusiv paginare si sortare.

Accesul la date este read-only pentru un anumit grup definit de utilizatori (partenerii RODA), 
iar pentru grupul de administratori este read-write. 
Desigur, nu sunt expuse anumite date sensibile precum hash-uri ale parolelor, drepturile de acces definite prin ACL etc.

\subsection{Adnotarea studiilor (termeni ELSST, cuvinte-cheie)}

ELSST (acronim = ???) este un thesaurus multi-lingual ce permite relatii ierarhice, sinonimii, definitii si relationarea termenilor.
??? Istoria ELSST ???

A fost realizat un servicu de importare a thesaurus-ului ELSST (in doua limbi, engleza si romana). 
Astfel, baza de date RODA este populata initial cu termenii si relatiile respective.
Mai multi astfel de termeni ELSST sunt asociati studiilor, si pot fi de asemenea asociati la nivelul seriilor de studii sau chiar al intrebarilor.
Intrucat incadrarea pe domenii tematice este un tip de informatie sensibila si pretioasa pentru cautari 
(inclusiv cele provenite din alte tari),
posibilitatea de a edita acest tip de informatii este rezervata doar grupului administratorilor.

Suplimentar, a fost prevazuta si implementata functionalitatea ca fiecare utilizator sa isi ataseze propriile cuvinte-cheie,
in mod nestructurat (oricate si oricare cuvinte-cheie, in orice limba) si independent de ceilalti utilizatori.

\subsection{Testarea aplicatiei}

??? detalii JUnit, Selenium, Jenkins

\clearpage

\section{Rezultate, stadiul realiz\u{a}rii obiectivului, concluzii si propuneri pentru continuarea proiectului}

%TODO de completat la fiecare faza ???


\medskip


%\clearpage

\bigskip

\bigskip

\bigskip

\bigskip

\bigskip

\bigskip

\bigskip

\bigskip

{\bfseries
RECTOR,}

prof.univ.dr. Mircea Dumitru

\bigskip

\bigskip

\bigskip

\bigskip

\bigskip

\bigskip

\bigskip

\bigskip

{\bfseries
DIRECTOR GENERAL ADMINISTRATIV,}

ec. Adrian Albu

\bigskip

\bigskip

\bigskip

\bigskip

\bigskip

\bigskip

\bigskip

\bigskip

{\bfseries
RESPONSABIL PROIECT,}

lect.univ.dr. Adrian Du\c{s}a

\end{document}
