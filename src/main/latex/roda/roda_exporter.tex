\section{Exportarea ca fisiere DDI}
Aplicatiei software a RODA i-a fost utila o componenta care sa exporte datele din baza de date a RODA in formatul DDI (format care evolueaza in timp, fiind poate necesara re-exportarea lor periodica).
Componenta de export poate exporta datele din tabelele bazei de date a RODA in doua formate:
\begin{itemize}
\item DDI-Codebook 1.2.2
\item DDI-Codebook 2.5
\end{itemize}

Procesul de export al fisierelor DDI este, ca schema de principiu, unul invers celui de import.
Pentru serializarea datelor in format XML compatibil cu schemele DDI (exprimate in XML Schema) 
am folosit una din solutiile OXM (Object-to-XML Mapping) ce sunt integrate in platforma Spring, respectiv componenta JAXB (Java-to-XML Binding, \url{https://jaxb.java.net/}).
Am utilizand din nou, ca si la import, pachetul 'ro.roda.ddi' (cuprinzand clase Java generate din XML Schema a standardului DDI), de data aceasta pentru a serializa obiectele in XML. 
Configurarea componentei de export se face prin fisierele 'applicationContext.xml' si 'roda.properties', iar pachetul care contine exporterul este 'ro.roda.exporter'.

\section{Exportarea altor tipuri de fisiere}
Celelate tipuri de fisiere utilizate pentru popularea bazei de date (toate cele in afara de fisierele DDI) sunt deja in format binar sau text/CSV, 
fiind salvate in repository-ul 'FileStore' / JCR ca date aferente la importarea de studii in format DDI.

Pentru componenta JCR (JackRabbit), 
se poate folosi pentru exportarea selectiva a datelor de tip fisiere utilitarul numit 'JcrImportExportTool' 
(care poate face filtrare dupa tipul de noduri, namespace-uri, versiuni etc.).