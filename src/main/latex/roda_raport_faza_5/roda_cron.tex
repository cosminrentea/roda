
\section{Task-Scheduling in Spring}

Spring 3.0 introduces a TaskScheduler with a variety of methods for scheduling tasks to run at some point in the future.
\begin{lstlisting}
public interface TaskScheduler {
ScheduledFuture schedule(Runnable task, Trigger trigger);

ScheduledFuture schedule(Runnable task, Date startTime);

ScheduledFuture scheduleAtFixedRate(Runnable task, Date startTime, long period);

ScheduledFuture scheduleAtFixedRate(Runnable task, long period);

ScheduledFuture scheduleWithFixedDelay(Runnable task, Date startTime, long delay);

ScheduledFuture scheduleWithFixedDelay(Runnable task, long delay);
}
\end{lstlisting}

The simplest method is the one named 'schedule' that takes a Runnable and Date only. 
That will cause the task to run once after the specified time. All of the other methods are capable of scheduling tasks to run repeatedly. 
The fixed-rate and fixed-delay methods are for simple, periodic execution, but the method that accepts a Trigger is much more flexible.

The Trigger interface is essentially inspired by JSR-236, which, as of Spring 3.0, has not yet been officially implemented. 
The basic idea of the Trigger is that execution times may be determined based on past execution outcomes or even arbitrary conditions. 
If these determinations do take into account the outcome of the preceding execution, that information is available within a TriggerContext.

Spring provides two implementations of the Trigger interface. The most interesting one is the CronTrigger. 
It enables the scheduling of tasks based on cron expressions. 

For example the following task is being scheduled to run 15 minutes past each hour but only during the 9-to-5 "business hours" on weekdays:
\begin{lstlisting}
scheduler.schedule(task, new CronTrigger("* 15 9-17 * * MON-FRI"));
\end{lstlisting}

The other out-of-the-box implementation is a PeriodicTrigger that accepts a fixed period, an optional initial delay value, 
and a boolean to indicate whether the period should be interpreted as a fixed-rate or a fixed-delay. 
Since the TaskScheduler interface already defines methods for scheduling tasks at a fixed-rate or with a fixed-delay, 
those methods should be used directly whenever possible. 
The value of the PeriodicTrigger implementation is that it can be used within components that rely on the Trigger abstraction. 
For example, it may be convenient to allow periodic triggers, cron-based triggers, and even custom trigger implementations to be used interchangeably. 
Such a component could take advantage of dependency injection so that such Triggers could be configured externally.

\section{Task-Scheduling in aplicatia RODA}

In fisierul 'applicationContext.xml' este configurata componenta de scheduling (avand un pool de 5 thread-uri):

\begin{lstlisting}
<task:scheduler id="scheduler" pool-size="5"/>
\end{lstlisting}

Pachetul in care se afla componenta de scheduling este 'ro.roda.scheduler', iar task-urile definite sunt in pachetul 'ro.roda.scheduler.tasks'.
Task-urile sunt implementari ale interfetelor Runnable sau RunnableFuture. 
Executiile task-urilor pot fi programate ori declansate manual.

Definitiile task-urilor sunt persistente in baza de date (si nu in fisiere de configuratie).
Tabelul ce contine task-urile posibile se populeaza sau actualizeaza la pornirea aplicatiei, 
dupa identificarea prin Java Reflection a tuturor task-urilor din pachetul mentionat.

De asemenea, si detaliile despre executiile aferente unui anumit task sunt stocate in baza de date.

Campurile acestor entitati sunt:
\begin{enumerate}
\item Task:
\begin{itemize}
\item id
\item name: numele task-ului
\item description: descrierea pe larg
\item classname: Numele complet al clasei Java care implementeaza task-ul
\item cron: string-ul in stil Cron care descrie regula de programare a executiei Task-ului; 
poate cuprinde campuri simple sau cu reguli pe baza de frecvente si intervale pentru: secunde, minute, ore, zile ale lunii, luni, zi a saptamanii, an.
\item enabled: daca task-ul este activat (sau dezactivat)
\item timestamp\_next\_execution: data si ora la care Task-ul este planificat pentru a fi lansat in executie (momentul planificat al urmatoarei executii)
\end{itemize}
\item Executie:
\begin{itemize}
\item id
\item task\_id : referinta la Task
\item type: "scheduled" sau "manual" (daca executia a fost programata sau a fost declansata manual)
\item result: rezultatul executiei (numar intreg, unde 0 reprezinta succes)
\item timestamp\_start: Momentul la care a inceput aceasta executie a task-ului
\item duration: durata acestei executii (in milisecunde)
\item stacktrace:  detalii in caz de executie fara succes
\end{itemize}
\end{enumerate}

Controlul sistemului de Task-Scheduling se realizeaza prin intermediul unor controllere care primesc comenzi
\begin{itemize}
\item de modificare a parametrilor task-urilor (dez-activare, modificare parametru stil Cron etc.)
\item de declansare manuala a executiei unui anumit task
\item de listare a tuturor task-urilor sau a unui anumit task
\item de listare a tuturor executiilor sau a unei anumite executii (cu detalii sau fara)
\item de listare filtrata a executiilor (de ex. dupa rezultat, sau dupa intervalul de timp in care a fost declansata, sau dupa durata)
\end{itemize}