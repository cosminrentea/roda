Datele initiale ale aplicatiei sunt importate chiar la pornirea serverului folosind un 'hook' al sistemului Spring MVC (ApplicationListener <ContextRefreshedEvent>).

Functionalitatile de importare a datelor sunt grupate intr-un singur serviciu (interfata si implementare) aflat in pachetul 'ro.roda.importer'.

Serviciul de import permite:
\begin{description}
\item
[Importarea datelor initiale ale aplicatiei] 

Aceste date cele ce vor fi folosite si in versiunea de productie a aplicatiei, precum: utilizatori si roluri, drepturi de acces standard, limbi, date geografice, thesaurus etc.
Se importa din mai multe formate:
\begin{itemize}
\item din format CSV (Comma-Separated Values). Se folosesc fie pachetul OpenCSV, fie API-ul oferit de driverul JDBC al Postgresql pentru a importa fisiere CSV).
\item din format SQL. Se foloseste fisierul 'import.sql', pe care componenta ORM (Hibernate) este configurata sa il importe la initializarea sa.
\item din thesaurus-ul ELSST. Se importa termenii si ulterior relatiile dintre ei din mai multe fisiere in format CSV, 
populandu-se mai multe tabele (cele din zona 'Topics' din schema bazei de date).
\item din formatul DDI-Codebook (Data Documentation Initiative, \url{http://www.ddialliance.org/Specification/DDI-Codebook/}).
\end{itemize}

\item
[Importarea datelor suplimentare / de testare pentru aplicatie.]

Aceste date sunt in format CSV. 
Sunt populate cu date mai multe tabele mai putin importante, care nu trebuie populate si in versiunea de productie, ci doar in cea de development si testare. 
\end{description}

Importer-ul din formatul DDI este realizat doar partial in aceasta faza a proiectului.
Se va folosi un pachet de clase Java derivat de catre JAXB de la definitia XML Schema a formatului DDI.
Aceasta componenta software va permite utilizarea integrala a datelor de tip legacy din arhiva electronica a RODA (50 studii sociale).
