\section{Task-Scheduling in Spring}

Spring 3.0 introduce un TaskScheduler cu o varietate de metode pentru programarea task-urilor pentru a rula la un anumit moment in viitor.
\begin{lstlisting}[breaklines=true]
public interface TaskScheduler {
ScheduledFuture schedule(Runnable task, Trigger trigger);

ScheduledFuture schedule(Runnable task, Date startTime);

ScheduledFuture scheduleAtFixedRate(Runnable task, Date startTime, long period);

ScheduledFuture scheduleAtFixedRate(Runnable task, long period);

ScheduledFuture scheduleWithFixedDelay(Runnable task, Date startTime, long delay);

ScheduledFuture scheduleWithFixedDelay(Runnable task, long delay);
}
\end{lstlisting}

Cea mai simpla metoda este cea numita 'schedule' care primeste doar un Runnable (task de rulat) si o Data (moment de timp).
Aceasta face ca task-ul sa ruleze o singura data dupa trecerea timpului specificat.
Toate celelalte metode sunt capabile sa programeze task-urile pentru a rula in mod repetat.
Metodele '...FixedRate' si '...FixedDelay' sunt pentru executii simple periodice, dar metoda care accepta un Trigger este mult mai flexibila.

Interfata Trigger este in esenta inspirata de catre JSR-236 (care nu fusese implementata oficial la momentul lansarii Spring 3.0).
Ideea de baza este accea ca momentele de executie pot fi determinate de rezultatele executiilor din trecut sau chiar alte conditii arbirare.
Daca aceste constrangeri iau in calcul rezultatele executiilor precedente, aceasta informatie e disponibila in cadrul unui TriggerContext.

Spring ofera doua implementari ale interfetei Trigger, cea mai interesanta dintre ele fiind CronTrigger, care permite programarea task-urilor pe baza unor expresii in stil 'cron'.
De exemplu, urmatorul task este programat sa ruleze la minutul 15 al fiecarei ore dar doar in intervalul orar 9-17 in timpul zilelor de lucru ale saptamanii:
\begin{lstlisting}
scheduler.schedule(task, new CronTrigger("* 15 9-17 * * MON-FRI"));
\end{lstlisting}

Cealalta implementare oferita este PeriodicTrigger, care accepta o perioada fixa, o intarziere initiala (optionala), 
si un flag boolean care sa indice daca perioada trebuie interpretata ca fiind 'fixed-rate' sau 'fixed-delay'.
Deoarece interfata TaskScheduler ofera deja metode pentru programarea task-urilor la o rata fixa sau cu o intarziere fixata intre executiile lor,
acele metode ar trebui folosite in mod direct atunci cand se poate.
Valoarea adusa de implementarea PeriodicTrigger este ca poate fi utilizat in cadrul unor componente care se bazeaza pe conceptul abstract de Trigger
(de exemplu, poate fi avantajos sa fie permisa utilizarea intersanjabila de triggere periodice, triggere bazate pe cron, sau chiar implementari customizate de Triggere;
o asemenea componenta poate beneficia de avantajul injectiei de dependente oferite de Spring, pentru a configura din exterior astfel de Triggere).

\section{Task-Scheduling in aplicatia RODA}

In fisierul 'applicationContext.xml' este configurata componenta de scheduling (avand un pool de 5 thread-uri):

\begin{lstlisting}
<task:scheduler id="scheduler" pool-size="5"/>
\end{lstlisting}

Pachetul in care se afla componenta de scheduling este 'ro.roda.scheduler', iar task-urile definite sunt in pachetul 'ro.roda.scheduler.tasks'.
Task-urile sunt implementari ale interfetelor Runnable sau RunnableFuture. 
Executiile task-urilor pot fi programate ori declansate manual.

Definitiile task-urilor sunt persistente in baza de date (si nu in fisiere de configuratie).
Tabelul ce contine task-urile posibile se populeaza sau actualizeaza la pornirea aplicatiei, 
dupa identificarea prin Java Reflection a tuturor task-urilor din pachetul mentionat.

De asemenea, si detaliile despre executiile aferente unui anumit task sunt stocate in baza de date.

Campurile acestor entitati sunt:
\begin{enumerate}
\item Task:
\begin{itemize}
\item id
\item name: numele task-ului
\item description: descrierea pe larg
\item classname: Numele complet al clasei Java care implementeaza task-ul
\item cron: string-ul in stil Cron care descrie regula de programare a executiei Task-ului; 
poate cuprinde campuri simple sau cu reguli pe baza de frecvente si intervale pentru: secunde, minute, ore, zile ale lunii, luni, zi a saptamanii, an.
\item enabled: daca task-ul este activat (sau dezactivat)
\item timestamp\_next\_execution: data si ora la care Task-ul este planificat pentru a fi lansat in executie (momentul planificat al urmatoarei executii)
\end{itemize}
\item Executie:
\begin{itemize}
\item id
\item task\_id : referinta la Task
\item type: "scheduled" sau "manual" (daca executia a fost programata sau a fost declansata manual)
\item result: rezultatul executiei (numar intreg, unde 0 reprezinta succes)
\item timestamp\_start: Momentul la care a inceput aceasta executie a task-ului
\item duration: durata acestei executii (in milisecunde)
\item stacktrace:  detalii in caz de executie fara succes
\end{itemize}
\end{enumerate}

Controlul sistemului de Task-Scheduling se realizeaza prin intermediul unor controllere care primesc comenzi
\begin{itemize}
\item de modificare a parametrilor task-urilor (dez-activare, modificare parametru stil Cron etc.)
\item de declansare manuala a executiei unui anumit task
\item de listare a tuturor task-urilor sau a unui anumit task
\item de listare a tuturor executiilor sau a unei anumite executii (cu detalii sau fara)
\item de listare filtrata a executiilor (de ex. dupa rezultat, sau dupa intervalul de timp in care a fost declansata, sau dupa durata)
\end{itemize}